\section{Related work}
\label{related}

We are aware of at least two attempts to formalize the static semantics
of Haskell, but neither of them fully specifies the module system.  In
the static semantics by Peyton Jones and Wadler \cite{SPJWad92}, the 
specification of imports was left as future work.  The authors rated it as
one of the highest-priority items on their todo list.  More recently,
Fax�n also worked on the static semantics of Haskell \cite{Faxen02}.  
In this work, he gave semantics to some parts of the module system, 
but also deviated from the report, opting for what he considered to be a
simpler (although non Haskell 98 compatible) specification 
(Section 4.2, \cite{Faxen02}).  Fax�n's work is consistent with the
report in that it does not specify how to treat mutually recursive
modules.

Our specification of the Haskell module system is relatively independent
of Haskell itself.  In this respect it is similar to Leroy's work 
on ML's module system \cite{Leroy-modular-modules}.
There have been numerous studies on advanced module systems, and the use 
of type theory to formalize them \cite{harper+:sharing,MacQueen:86}.
In the same spirit, there has recently been a proposal
for a replacement of the Haskell module system by Shields and Peyton Jones
\cite{shields-pj:first-class-modules}.




