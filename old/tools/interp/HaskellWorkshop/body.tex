\documentclass{entcs} \usepackage{entcsmacro}
\usepackage{graphicx}

% Psfig/TeX Release 1.3
% dvips version
%
% All software, documentation, and related files in this distribution of
% psfig/tex are Copyright 1987, 1988 Trevor J. Darrell
%
% Permission is granted for use and non-profit distribution of psfig/tex 
% providing that this notice be clearly maintained, but the right to
% distribute any portion of psfig/tex for profit or as part of any commercial
% product is specifically reserved for the author.
%
% $Header: /cvs/programatica/tools/interp/HaskellWorkshop/psfig.tex,v 1.1 2001/04/25 23:52:51 wlh Exp $
% $Source: /cvs/programatica/tools/interp/HaskellWorkshop/psfig.tex,v $
%
% Thanks to Greg Hager (GDH) and Ned Batchelder for their contributions
% to this project.
%
% Modified by J. Daniel Smith on 9 October 1990 to accept the
% %%BoundingBox: comment with or without a space after the colon.  Stole
% file reading code from Tom Rokicki's EPSF.TEX file (see below).

%
% from a suggestion by eijkhout@csrd.uiuc.edu to allow
% loading as a style file:
\edef\psfigRestoreAt{\catcode`@=\number\catcode`@\relax}
\catcode`\@=11\relax
\newwrite\@unused
\def\typeout#1{{\let\protect\string\immediate\write\@unused{#1}}}
\typeout{psfig/tex 1.4-dvips}


%% Here's how you define your figure path.  Should be set up with null
%% default and a user useable definition.

\def\figurepath{./}
\def\psfigurepath#1{\edef\figurepath{#1}}

%
% @psdo control structure -- similar to Latex @for.
% I redefined these with different names so that psfig can
% be used with TeX as well as LaTeX, and so that it will not 
% be vunerable to future changes in LaTeX's internal
% control structure,
%
\def\@nnil{\@nil}
\def\@empty{}
\def\@psdonoop#1\@@#2#3{}
\def\@psdo#1:=#2\do#3{\edef\@psdotmp{#2}\ifx\@psdotmp\@empty \else
    \expandafter\@psdoloop#2,\@nil,\@nil\@@#1{#3}\fi}
\def\@psdoloop#1,#2,#3\@@#4#5{\def#4{#1}\ifx #4\@nnil \else
       #5\def#4{#2}\ifx #4\@nnil \else#5\@ipsdoloop #3\@@#4{#5}\fi\fi}
\def\@ipsdoloop#1,#2\@@#3#4{\def#3{#1}\ifx #3\@nnil 
       \let\@nextwhile=\@psdonoop \else
      #4\relax\let\@nextwhile=\@ipsdoloop\fi\@nextwhile#2\@@#3{#4}}
\def\@tpsdo#1:=#2\do#3{\xdef\@psdotmp{#2}\ifx\@psdotmp\@empty \else
    \@tpsdoloop#2\@nil\@nil\@@#1{#3}\fi}
\def\@tpsdoloop#1#2\@@#3#4{\def#3{#1}\ifx #3\@nnil 
       \let\@nextwhile=\@psdonoop \else
      #4\relax\let\@nextwhile=\@tpsdoloop\fi\@nextwhile#2\@@#3{#4}}
% 
%
%%%%%%%%%%%%%%%%%%%%%%%%%%%%%%%%%%%%%%%%%%%%%%%%%%%%%%%%%%%%%%%%%%%
% file reading stuff from epsf.tex
%   EPSF.TEX macro file:
%   Written by Tomas Rokicki of Radical Eye Software, 29 Mar 1989.
%   Revised by Don Knuth, 3 Jan 1990.
%   Revised by Tomas Rokicki to accept bounding boxes with no
%      space after the colon, 18 Jul 1990.
%   Portions modified/removed for use in PSFIG package by
%      J. Daniel Smith, 9 October 1990.
%
\newread\ps@stream
\newif\ifnot@eof       % continue looking for the bounding box?
\newif\if@noisy        % report what you're making?
\newif\if@atend        % %%BoundingBox: has (at end) specification
\newif\if@psfile       % does this look like a PostScript file?
%
% PostScript files should start with `%!'
%
{\catcode`\%=12\global\gdef\epsf@start{%!}}
\def\epsf@PS{PS}
%
\def\epsf@getbb#1{%
%
%   The first thing we need to do is to open the
%   PostScript file, if possible.
%
\openin\ps@stream=#1
\ifeof\ps@stream\typeout{Error, File #1 not found}\else
%
%   Okay, we got it. Now we'll scan lines until we find one that doesn't
%   start with %. We're looking for the bounding box comment.
%
   {\not@eoftrue \chardef\other=12
    \def\do##1{\catcode`##1=\other}\dospecials \catcode`\ =10
    \loop
       \if@psfile
	  \read\ps@stream to \epsf@fileline
       \else{
	  \obeyspaces
          \read\ps@stream to \epsf@tmp\global\let\epsf@fileline\epsf@tmp}
       \fi
       \ifeof\ps@stream\not@eoffalse\else
%
%   Check the first line for `%!'.  Issue a warning message if its not
%   there, since the file might not be a PostScript file.
%
       \if@psfile\else
       \expandafter\epsf@test\epsf@fileline:. \\%
       \fi
%
%   We check to see if the first character is a % sign;
%   if so, we look further and stop only if the line begins with
%   `%%BoundingBox:' and the `(atend)' specification was not found.
%   That is, the only way to stop is when the end of file is reached,
%   or a `%%BoundingBox: llx lly urx ury' line is found.
%
          \expandafter\epsf@aux\epsf@fileline:. \\%
       \fi
   \ifnot@eof\repeat
   }\closein\ps@stream\fi}%
%
% This tests if the file we are reading looks like a PostScript file.
%
\long\def\epsf@test#1#2#3:#4\\{\def\epsf@testit{#1#2}
			\ifx\epsf@testit\epsf@start\else
\typeout{Warning! File does not start with `\epsf@start'.  It may not be a PostScript file.}
			\fi
			\@psfiletrue} % don't test after 1st line
%
%   We still need to define the tricky \epsf@aux macro. This requires
%   a couple of magic constants for comparison purposes.
%
{\catcode`\%=12\global\let\epsf@percent=%\global\def\epsf@bblit{%BoundingBox}}
%
%
%   So we're ready to check for `%BoundingBox:' and to grab the
%   values if they are found.  We continue searching if `(at end)'
%   was found after the `%BoundingBox:'.
%
\long\def\epsf@aux#1#2:#3\\{\ifx#1\epsf@percent
   \def\epsf@testit{#2}\ifx\epsf@testit\epsf@bblit
	\@atendfalse
        \epsf@atend #3 . \\%
	\if@atend	
	   \if@verbose{
		\typeout{psfig: found `(atend)'; continuing search}
	   }\fi
        \else
        \epsf@grab #3 . . . \\%
        \not@eoffalse
        \global\no@bbfalse
        \fi
   \fi\fi}%
%
%   Here we grab the values and stuff them in the appropriate definitions.
%
\def\epsf@grab #1 #2 #3 #4 #5\\{%
   \global\def\epsf@llx{#1}\ifx\epsf@llx\empty
      \epsf@grab #2 #3 #4 #5 .\\\else
   \global\def\epsf@lly{#2}%
   \global\def\epsf@urx{#3}\global\def\epsf@ury{#4}\fi}%
%
% Determine if the stuff following the %%BoundingBox is `(atend)'
% J. Daniel Smith.  Copied from \epsf@grab above.
%
\def\epsf@atendlit{(atend)} 
\def\epsf@atend #1 #2 #3\\{%
   \def\epsf@tmp{#1}\ifx\epsf@tmp\empty
      \epsf@atend #2 #3 .\\\else
   \ifx\epsf@tmp\epsf@atendlit\@atendtrue\fi\fi}


% End of file reading stuff from epsf.tex
%%%%%%%%%%%%%%%%%%%%%%%%%%%%%%%%%%%%%%%%%%%%%%%%%%%%%%%%%%%%%%%%%%%

\def\psdraft{
	\def\@psdraft{0}
	%\typeout{draft level now is \@psdraft \space . }
}
\def\psfull{
	\def\@psdraft{100}
	%\typeout{draft level now is \@psdraft \space . }
}
\psfull
\newif\if@prologfile
\newif\if@postlogfile
\def\pssilent{
	\@noisyfalse
}
\def\psnoisy{
	\@noisytrue
}
\psnoisy
%%% These are for the option list.
%%% A specification of the form a = b maps to calling \@p@@sa{b}
\newif\if@bbllx
\newif\if@bblly
\newif\if@bburx
\newif\if@bbury
\newif\if@height
\newif\if@width
\newif\if@rheight
\newif\if@rwidth
\newif\if@clip
\newif\if@verbose
\def\@p@@sclip#1{\@cliptrue}

%%% GDH 7/26/87 -- changed so that it first looks in the local directory,
%%% then in a specified global directory for the ps file.

\def\@p@@sfile#1{\def\@p@sfile{null}%
	        \openin1=#1
		\ifeof1\closein1%
		       \openin1=\figurepath#1
			\ifeof1\typeout{Error, File #1 not found}
			\else\closein1
			    \edef\@p@sfile{\figurepath#1}%
                        \fi%
		 \else\closein1%
		       \def\@p@sfile{#1}%
		 \fi}
\def\@p@@sfigure#1{\def\@p@sfile{null}%
	        \openin1=#1
		\ifeof1\closein1%
		       \openin1=\figurepath#1
			\ifeof1\typeout{Error, File #1 not found}
			\else\closein1
			    \def\@p@sfile{\figurepath#1}%
                        \fi%
		 \else\closein1%
		       \def\@p@sfile{#1}%
		 \fi}

\def\@p@@sbbllx#1{
		%\typeout{bbllx is #1}
		\@bbllxtrue
		\dimen100=#1
		\edef\@p@sbbllx{\number\dimen100}
}
\def\@p@@sbblly#1{
		%\typeout{bblly is #1}
		\@bbllytrue
		\dimen100=#1
		\edef\@p@sbblly{\number\dimen100}
}
\def\@p@@sbburx#1{
		%\typeout{bburx is #1}
		\@bburxtrue
		\dimen100=#1
		\edef\@p@sbburx{\number\dimen100}
}
\def\@p@@sbbury#1{
		%\typeout{bbury is #1}
		\@bburytrue
		\dimen100=#1
		\edef\@p@sbbury{\number\dimen100}
}
\def\@p@@sheight#1{
		\@heighttrue
		\dimen100=#1
   		\edef\@p@sheight{\number\dimen100}
		%\typeout{Height is \@p@sheight}
}
\def\@p@@swidth#1{
		%\typeout{Width is #1}
		\@widthtrue
		\dimen100=#1
		\edef\@p@swidth{\number\dimen100}
}
\def\@p@@srheight#1{
		%\typeout{Reserved height is #1}
		\@rheighttrue
		\dimen100=#1
		\edef\@p@srheight{\number\dimen100}
}
\def\@p@@srwidth#1{
		%\typeout{Reserved width is #1}
		\@rwidthtrue
		\dimen100=#1
		\edef\@p@srwidth{\number\dimen100}
}
\def\@p@@ssilent#1{ 
		\@verbosefalse
}
\def\@p@@sprolog#1{\@prologfiletrue\def\@prologfileval{#1}}
\def\@p@@spostlog#1{\@postlogfiletrue\def\@postlogfileval{#1}}
\def\@cs@name#1{\csname #1\endcsname}
\def\@setparms#1=#2,{\@cs@name{@p@@s#1}{#2}}
%
% initialize the defaults (size the size of the figure)
%
\def\ps@init@parms{
		\@bbllxfalse \@bbllyfalse
		\@bburxfalse \@bburyfalse
		\@heightfalse \@widthfalse
		\@rheightfalse \@rwidthfalse
		\def\@p@sbbllx{}\def\@p@sbblly{}
		\def\@p@sbburx{}\def\@p@sbbury{}
		\def\@p@sheight{}\def\@p@swidth{}
		\def\@p@srheight{}\def\@p@srwidth{}
		\def\@p@sfile{}
		\def\@p@scost{10}
		\def\@sc{}
		\@prologfilefalse
		\@postlogfilefalse
		\@clipfalse
		\if@noisy
			\@verbosetrue
		\else
			\@verbosefalse
		\fi
}
%
% Go through the options setting things up.
%
\def\parse@ps@parms#1{
	 	\@psdo\@psfiga:=#1\do
		   {\expandafter\@setparms\@psfiga,}}
%
% Compute bb height and width
%
\newif\ifno@bb
\def\bb@missing{
	\if@verbose{
		\typeout{psfig: searching \@p@sfile \space  for bounding box}
	}\fi
	\no@bbtrue
	\epsf@getbb{\@p@sfile}
        \ifno@bb \else \bb@cull\epsf@llx\epsf@lly\epsf@urx\epsf@ury\fi
}	
\def\bb@cull#1#2#3#4{
	\dimen100=#1 bp\edef\@p@sbbllx{\number\dimen100}
	\dimen100=#2 bp\edef\@p@sbblly{\number\dimen100}
	\dimen100=#3 bp\edef\@p@sbburx{\number\dimen100}
	\dimen100=#4 bp\edef\@p@sbbury{\number\dimen100}
	\no@bbfalse
}
\def\compute@bb{
		\no@bbfalse
		\if@bbllx \else \no@bbtrue \fi
		\if@bblly \else \no@bbtrue \fi
		\if@bburx \else \no@bbtrue \fi
		\if@bbury \else \no@bbtrue \fi
		\ifno@bb \bb@missing \fi
		\ifno@bb \typeout{FATAL ERROR: no bb supplied or found}
			\no-bb-error
		\fi
		%
		\count203=\@p@sbburx
		\count204=\@p@sbbury
		\advance\count203 by -\@p@sbbllx
		\advance\count204 by -\@p@sbblly
		\edef\@bbw{\number\count203}
		\edef\@bbh{\number\count204}
		%\typeout{ bbh = \@bbh, bbw = \@bbw }
}
%
% \in@hundreds performs #1 * (#2 / #3) correct to the hundreds,
%	then leaves the result in @result
%
\def\in@hundreds#1#2#3{\count240=#2 \count241=#3
		     \count100=\count240	% 100 is first digit #2/#3
		     \divide\count100 by \count241
		     \count101=\count100
		     \multiply\count101 by \count241
		     \advance\count240 by -\count101
		     \multiply\count240 by 10
		     \count101=\count240	%101 is second digit of #2/#3
		     \divide\count101 by \count241
		     \count102=\count101
		     \multiply\count102 by \count241
		     \advance\count240 by -\count102
		     \multiply\count240 by 10
		     \count102=\count240	% 102 is the third digit
		     \divide\count102 by \count241
		     \count200=#1\count205=0
		     \count201=\count200
			\multiply\count201 by \count100
		 	\advance\count205 by \count201
		     \count201=\count200
			\divide\count201 by 10
			\multiply\count201 by \count101
			\advance\count205 by \count201
			%
		     \count201=\count200
			\divide\count201 by 100
			\multiply\count201 by \count102
			\advance\count205 by \count201
			%
		     \edef\@result{\number\count205}
}
\def\compute@wfromh{
		% computing : width = height * (bbw / bbh)
		\in@hundreds{\@p@sheight}{\@bbw}{\@bbh}
		%\typeout{ \@p@sheight * \@bbw / \@bbh, = \@result }
		\edef\@p@swidth{\@result}
		%\typeout{w from h: width is \@p@swidth}
}
\def\compute@hfromw{
		% computing : height = width * (bbh / bbw)
		\in@hundreds{\@p@swidth}{\@bbh}{\@bbw}
		%\typeout{ \@p@swidth * \@bbh / \@bbw = \@result }
		\edef\@p@sheight{\@result}
		%\typeout{h from w : height is \@p@sheight}
}
\def\compute@handw{
		\if@height 
			\if@width
			\else
				\compute@wfromh
			\fi
		\else 
			\if@width
				\compute@hfromw
			\else
				\edef\@p@sheight{\@bbh}
				\edef\@p@swidth{\@bbw}
			\fi
		\fi
}
\def\compute@resv{
		\if@rheight \else \edef\@p@srheight{\@p@sheight} \fi
		\if@rwidth \else \edef\@p@srwidth{\@p@swidth} \fi
}
%		
% Compute any missing values
\def\compute@sizes{
	\compute@bb
	\compute@handw
	\compute@resv
}
%
% \psfig
% usage : \psfig{file=, height=, width=, bbllx=, bblly=, bburx=, bbury=,
%			rheight=, rwidth=, clip=}
%
% "clip=" is a switch and takes no value, but the `=' must be present.
\def\psfig#1{\vbox {
	% do a zero width hard space so that a single
	% \psfig in a centering enviornment will behave nicely
	%{\setbox0=\hbox{\ }\ \hskip-\wd0}
	%
	\ps@init@parms
	\parse@ps@parms{#1}
	\compute@sizes
	%
	\ifnum\@p@scost<\@psdraft{
		\if@verbose{
			\typeout{psfig: including \@p@sfile \space }
		}\fi
		%
		\special{ps::[begin] 	\@p@swidth \space \@p@sheight \space
				\@p@sbbllx \space \@p@sbblly \space
				\@p@sbburx \space \@p@sbbury \space
				startTexFig \space }
		\if@clip{
			\if@verbose{
				\typeout{(clip)}
			}\fi
			\special{ps:: doclip \space }
		}\fi
		\if@prologfile
		    \special{ps: plotfile \@prologfileval \space } \fi
		\special{ps: plotfile \@p@sfile \space }
		\if@postlogfile
		    \special{ps: plotfile \@postlogfileval \space } \fi
		\special{ps::[end] endTexFig \space }
		% Create the vbox to reserve the space for the figure
		\vbox to \@p@srheight true sp{
			\hbox to \@p@srwidth true sp{
				\hss
			}
		\vss
		}
	}\else{
		% draft figure, just reserve the space and print the
		% path name.
		\vbox to \@p@srheight true sp{
		\vss
			\hbox to \@p@srwidth true sp{
				\hss
				\if@verbose{
					\@p@sfile
				}\fi
				\hss
			}
		\vss
		}
	}\fi
}}
\def\psglobal{\typeout{psfig: PSGLOBAL is OBSOLETE; use psprint -m instead}}
\psfigRestoreAt






				% <---- load psfig macros
\newcommand{\psinput}[1]{
  \mbox{\psfig{figure=#1}}}
\newcommand{\hgap}{\hspace{12pt}}
\newcommand{\vgap}{\\[12pt]}
\newcommand{\pscap}[2]{
  \parbox{2.01in}{
   \begin{center}
    \psinput{#1} \\ #2
   \end{center}
  }
}

% Psfig/TeX Release 1.3
% dvips version
%
% All software, documentation, and related files in this distribution of
% psfig/tex are Copyright 1987, 1988 Trevor J. Darrell
%
% Permission is granted for use and non-profit distribution of psfig/tex 
% providing that this notice be clearly maintained, but the right to
% distribute any portion of psfig/tex for profit or as part of any commercial
% product is specifically reserved for the author.
%
% $Header: /cvs/programatica/tools/interp/HaskellWorkshop/psfig.tex,v 1.1 2001/04/25 23:52:51 wlh Exp $
% $Source: /cvs/programatica/tools/interp/HaskellWorkshop/psfig.tex,v $
%
% Thanks to Greg Hager (GDH) and Ned Batchelder for their contributions
% to this project.
%
% Modified by J. Daniel Smith on 9 October 1990 to accept the
% %%BoundingBox: comment with or without a space after the colon.  Stole
% file reading code from Tom Rokicki's EPSF.TEX file (see below).

%
% from a suggestion by eijkhout@csrd.uiuc.edu to allow
% loading as a style file:
\edef\psfigRestoreAt{\catcode`@=\number\catcode`@\relax}
\catcode`\@=11\relax
\newwrite\@unused
\def\typeout#1{{\let\protect\string\immediate\write\@unused{#1}}}
\typeout{psfig/tex 1.4-dvips}


%% Here's how you define your figure path.  Should be set up with null
%% default and a user useable definition.

\def\figurepath{./}
\def\psfigurepath#1{\edef\figurepath{#1}}

%
% @psdo control structure -- similar to Latex @for.
% I redefined these with different names so that psfig can
% be used with TeX as well as LaTeX, and so that it will not 
% be vunerable to future changes in LaTeX's internal
% control structure,
%
\def\@nnil{\@nil}
\def\@empty{}
\def\@psdonoop#1\@@#2#3{}
\def\@psdo#1:=#2\do#3{\edef\@psdotmp{#2}\ifx\@psdotmp\@empty \else
    \expandafter\@psdoloop#2,\@nil,\@nil\@@#1{#3}\fi}
\def\@psdoloop#1,#2,#3\@@#4#5{\def#4{#1}\ifx #4\@nnil \else
       #5\def#4{#2}\ifx #4\@nnil \else#5\@ipsdoloop #3\@@#4{#5}\fi\fi}
\def\@ipsdoloop#1,#2\@@#3#4{\def#3{#1}\ifx #3\@nnil 
       \let\@nextwhile=\@psdonoop \else
      #4\relax\let\@nextwhile=\@ipsdoloop\fi\@nextwhile#2\@@#3{#4}}
\def\@tpsdo#1:=#2\do#3{\xdef\@psdotmp{#2}\ifx\@psdotmp\@empty \else
    \@tpsdoloop#2\@nil\@nil\@@#1{#3}\fi}
\def\@tpsdoloop#1#2\@@#3#4{\def#3{#1}\ifx #3\@nnil 
       \let\@nextwhile=\@psdonoop \else
      #4\relax\let\@nextwhile=\@tpsdoloop\fi\@nextwhile#2\@@#3{#4}}
% 
%
%%%%%%%%%%%%%%%%%%%%%%%%%%%%%%%%%%%%%%%%%%%%%%%%%%%%%%%%%%%%%%%%%%%
% file reading stuff from epsf.tex
%   EPSF.TEX macro file:
%   Written by Tomas Rokicki of Radical Eye Software, 29 Mar 1989.
%   Revised by Don Knuth, 3 Jan 1990.
%   Revised by Tomas Rokicki to accept bounding boxes with no
%      space after the colon, 18 Jul 1990.
%   Portions modified/removed for use in PSFIG package by
%      J. Daniel Smith, 9 October 1990.
%
\newread\ps@stream
\newif\ifnot@eof       % continue looking for the bounding box?
\newif\if@noisy        % report what you're making?
\newif\if@atend        % %%BoundingBox: has (at end) specification
\newif\if@psfile       % does this look like a PostScript file?
%
% PostScript files should start with `%!'
%
{\catcode`\%=12\global\gdef\epsf@start{%!}}
\def\epsf@PS{PS}
%
\def\epsf@getbb#1{%
%
%   The first thing we need to do is to open the
%   PostScript file, if possible.
%
\openin\ps@stream=#1
\ifeof\ps@stream\typeout{Error, File #1 not found}\else
%
%   Okay, we got it. Now we'll scan lines until we find one that doesn't
%   start with %. We're looking for the bounding box comment.
%
   {\not@eoftrue \chardef\other=12
    \def\do##1{\catcode`##1=\other}\dospecials \catcode`\ =10
    \loop
       \if@psfile
	  \read\ps@stream to \epsf@fileline
       \else{
	  \obeyspaces
          \read\ps@stream to \epsf@tmp\global\let\epsf@fileline\epsf@tmp}
       \fi
       \ifeof\ps@stream\not@eoffalse\else
%
%   Check the first line for `%!'.  Issue a warning message if its not
%   there, since the file might not be a PostScript file.
%
       \if@psfile\else
       \expandafter\epsf@test\epsf@fileline:. \\%
       \fi
%
%   We check to see if the first character is a % sign;
%   if so, we look further and stop only if the line begins with
%   `%%BoundingBox:' and the `(atend)' specification was not found.
%   That is, the only way to stop is when the end of file is reached,
%   or a `%%BoundingBox: llx lly urx ury' line is found.
%
          \expandafter\epsf@aux\epsf@fileline:. \\%
       \fi
   \ifnot@eof\repeat
   }\closein\ps@stream\fi}%
%
% This tests if the file we are reading looks like a PostScript file.
%
\long\def\epsf@test#1#2#3:#4\\{\def\epsf@testit{#1#2}
			\ifx\epsf@testit\epsf@start\else
\typeout{Warning! File does not start with `\epsf@start'.  It may not be a PostScript file.}
			\fi
			\@psfiletrue} % don't test after 1st line
%
%   We still need to define the tricky \epsf@aux macro. This requires
%   a couple of magic constants for comparison purposes.
%
{\catcode`\%=12\global\let\epsf@percent=%\global\def\epsf@bblit{%BoundingBox}}
%
%
%   So we're ready to check for `%BoundingBox:' and to grab the
%   values if they are found.  We continue searching if `(at end)'
%   was found after the `%BoundingBox:'.
%
\long\def\epsf@aux#1#2:#3\\{\ifx#1\epsf@percent
   \def\epsf@testit{#2}\ifx\epsf@testit\epsf@bblit
	\@atendfalse
        \epsf@atend #3 . \\%
	\if@atend	
	   \if@verbose{
		\typeout{psfig: found `(atend)'; continuing search}
	   }\fi
        \else
        \epsf@grab #3 . . . \\%
        \not@eoffalse
        \global\no@bbfalse
        \fi
   \fi\fi}%
%
%   Here we grab the values and stuff them in the appropriate definitions.
%
\def\epsf@grab #1 #2 #3 #4 #5\\{%
   \global\def\epsf@llx{#1}\ifx\epsf@llx\empty
      \epsf@grab #2 #3 #4 #5 .\\\else
   \global\def\epsf@lly{#2}%
   \global\def\epsf@urx{#3}\global\def\epsf@ury{#4}\fi}%
%
% Determine if the stuff following the %%BoundingBox is `(atend)'
% J. Daniel Smith.  Copied from \epsf@grab above.
%
\def\epsf@atendlit{(atend)} 
\def\epsf@atend #1 #2 #3\\{%
   \def\epsf@tmp{#1}\ifx\epsf@tmp\empty
      \epsf@atend #2 #3 .\\\else
   \ifx\epsf@tmp\epsf@atendlit\@atendtrue\fi\fi}


% End of file reading stuff from epsf.tex
%%%%%%%%%%%%%%%%%%%%%%%%%%%%%%%%%%%%%%%%%%%%%%%%%%%%%%%%%%%%%%%%%%%

\def\psdraft{
	\def\@psdraft{0}
	%\typeout{draft level now is \@psdraft \space . }
}
\def\psfull{
	\def\@psdraft{100}
	%\typeout{draft level now is \@psdraft \space . }
}
\psfull
\newif\if@prologfile
\newif\if@postlogfile
\def\pssilent{
	\@noisyfalse
}
\def\psnoisy{
	\@noisytrue
}
\psnoisy
%%% These are for the option list.
%%% A specification of the form a = b maps to calling \@p@@sa{b}
\newif\if@bbllx
\newif\if@bblly
\newif\if@bburx
\newif\if@bbury
\newif\if@height
\newif\if@width
\newif\if@rheight
\newif\if@rwidth
\newif\if@clip
\newif\if@verbose
\def\@p@@sclip#1{\@cliptrue}

%%% GDH 7/26/87 -- changed so that it first looks in the local directory,
%%% then in a specified global directory for the ps file.

\def\@p@@sfile#1{\def\@p@sfile{null}%
	        \openin1=#1
		\ifeof1\closein1%
		       \openin1=\figurepath#1
			\ifeof1\typeout{Error, File #1 not found}
			\else\closein1
			    \edef\@p@sfile{\figurepath#1}%
                        \fi%
		 \else\closein1%
		       \def\@p@sfile{#1}%
		 \fi}
\def\@p@@sfigure#1{\def\@p@sfile{null}%
	        \openin1=#1
		\ifeof1\closein1%
		       \openin1=\figurepath#1
			\ifeof1\typeout{Error, File #1 not found}
			\else\closein1
			    \def\@p@sfile{\figurepath#1}%
                        \fi%
		 \else\closein1%
		       \def\@p@sfile{#1}%
		 \fi}

\def\@p@@sbbllx#1{
		%\typeout{bbllx is #1}
		\@bbllxtrue
		\dimen100=#1
		\edef\@p@sbbllx{\number\dimen100}
}
\def\@p@@sbblly#1{
		%\typeout{bblly is #1}
		\@bbllytrue
		\dimen100=#1
		\edef\@p@sbblly{\number\dimen100}
}
\def\@p@@sbburx#1{
		%\typeout{bburx is #1}
		\@bburxtrue
		\dimen100=#1
		\edef\@p@sbburx{\number\dimen100}
}
\def\@p@@sbbury#1{
		%\typeout{bbury is #1}
		\@bburytrue
		\dimen100=#1
		\edef\@p@sbbury{\number\dimen100}
}
\def\@p@@sheight#1{
		\@heighttrue
		\dimen100=#1
   		\edef\@p@sheight{\number\dimen100}
		%\typeout{Height is \@p@sheight}
}
\def\@p@@swidth#1{
		%\typeout{Width is #1}
		\@widthtrue
		\dimen100=#1
		\edef\@p@swidth{\number\dimen100}
}
\def\@p@@srheight#1{
		%\typeout{Reserved height is #1}
		\@rheighttrue
		\dimen100=#1
		\edef\@p@srheight{\number\dimen100}
}
\def\@p@@srwidth#1{
		%\typeout{Reserved width is #1}
		\@rwidthtrue
		\dimen100=#1
		\edef\@p@srwidth{\number\dimen100}
}
\def\@p@@ssilent#1{ 
		\@verbosefalse
}
\def\@p@@sprolog#1{\@prologfiletrue\def\@prologfileval{#1}}
\def\@p@@spostlog#1{\@postlogfiletrue\def\@postlogfileval{#1}}
\def\@cs@name#1{\csname #1\endcsname}
\def\@setparms#1=#2,{\@cs@name{@p@@s#1}{#2}}
%
% initialize the defaults (size the size of the figure)
%
\def\ps@init@parms{
		\@bbllxfalse \@bbllyfalse
		\@bburxfalse \@bburyfalse
		\@heightfalse \@widthfalse
		\@rheightfalse \@rwidthfalse
		\def\@p@sbbllx{}\def\@p@sbblly{}
		\def\@p@sbburx{}\def\@p@sbbury{}
		\def\@p@sheight{}\def\@p@swidth{}
		\def\@p@srheight{}\def\@p@srwidth{}
		\def\@p@sfile{}
		\def\@p@scost{10}
		\def\@sc{}
		\@prologfilefalse
		\@postlogfilefalse
		\@clipfalse
		\if@noisy
			\@verbosetrue
		\else
			\@verbosefalse
		\fi
}
%
% Go through the options setting things up.
%
\def\parse@ps@parms#1{
	 	\@psdo\@psfiga:=#1\do
		   {\expandafter\@setparms\@psfiga,}}
%
% Compute bb height and width
%
\newif\ifno@bb
\def\bb@missing{
	\if@verbose{
		\typeout{psfig: searching \@p@sfile \space  for bounding box}
	}\fi
	\no@bbtrue
	\epsf@getbb{\@p@sfile}
        \ifno@bb \else \bb@cull\epsf@llx\epsf@lly\epsf@urx\epsf@ury\fi
}	
\def\bb@cull#1#2#3#4{
	\dimen100=#1 bp\edef\@p@sbbllx{\number\dimen100}
	\dimen100=#2 bp\edef\@p@sbblly{\number\dimen100}
	\dimen100=#3 bp\edef\@p@sbburx{\number\dimen100}
	\dimen100=#4 bp\edef\@p@sbbury{\number\dimen100}
	\no@bbfalse
}
\def\compute@bb{
		\no@bbfalse
		\if@bbllx \else \no@bbtrue \fi
		\if@bblly \else \no@bbtrue \fi
		\if@bburx \else \no@bbtrue \fi
		\if@bbury \else \no@bbtrue \fi
		\ifno@bb \bb@missing \fi
		\ifno@bb \typeout{FATAL ERROR: no bb supplied or found}
			\no-bb-error
		\fi
		%
		\count203=\@p@sbburx
		\count204=\@p@sbbury
		\advance\count203 by -\@p@sbbllx
		\advance\count204 by -\@p@sbblly
		\edef\@bbw{\number\count203}
		\edef\@bbh{\number\count204}
		%\typeout{ bbh = \@bbh, bbw = \@bbw }
}
%
% \in@hundreds performs #1 * (#2 / #3) correct to the hundreds,
%	then leaves the result in @result
%
\def\in@hundreds#1#2#3{\count240=#2 \count241=#3
		     \count100=\count240	% 100 is first digit #2/#3
		     \divide\count100 by \count241
		     \count101=\count100
		     \multiply\count101 by \count241
		     \advance\count240 by -\count101
		     \multiply\count240 by 10
		     \count101=\count240	%101 is second digit of #2/#3
		     \divide\count101 by \count241
		     \count102=\count101
		     \multiply\count102 by \count241
		     \advance\count240 by -\count102
		     \multiply\count240 by 10
		     \count102=\count240	% 102 is the third digit
		     \divide\count102 by \count241
		     \count200=#1\count205=0
		     \count201=\count200
			\multiply\count201 by \count100
		 	\advance\count205 by \count201
		     \count201=\count200
			\divide\count201 by 10
			\multiply\count201 by \count101
			\advance\count205 by \count201
			%
		     \count201=\count200
			\divide\count201 by 100
			\multiply\count201 by \count102
			\advance\count205 by \count201
			%
		     \edef\@result{\number\count205}
}
\def\compute@wfromh{
		% computing : width = height * (bbw / bbh)
		\in@hundreds{\@p@sheight}{\@bbw}{\@bbh}
		%\typeout{ \@p@sheight * \@bbw / \@bbh, = \@result }
		\edef\@p@swidth{\@result}
		%\typeout{w from h: width is \@p@swidth}
}
\def\compute@hfromw{
		% computing : height = width * (bbh / bbw)
		\in@hundreds{\@p@swidth}{\@bbh}{\@bbw}
		%\typeout{ \@p@swidth * \@bbh / \@bbw = \@result }
		\edef\@p@sheight{\@result}
		%\typeout{h from w : height is \@p@sheight}
}
\def\compute@handw{
		\if@height 
			\if@width
			\else
				\compute@wfromh
			\fi
		\else 
			\if@width
				\compute@hfromw
			\else
				\edef\@p@sheight{\@bbh}
				\edef\@p@swidth{\@bbw}
			\fi
		\fi
}
\def\compute@resv{
		\if@rheight \else \edef\@p@srheight{\@p@sheight} \fi
		\if@rwidth \else \edef\@p@srwidth{\@p@swidth} \fi
}
%		
% Compute any missing values
\def\compute@sizes{
	\compute@bb
	\compute@handw
	\compute@resv
}
%
% \psfig
% usage : \psfig{file=, height=, width=, bbllx=, bblly=, bburx=, bbury=,
%			rheight=, rwidth=, clip=}
%
% "clip=" is a switch and takes no value, but the `=' must be present.
\def\psfig#1{\vbox {
	% do a zero width hard space so that a single
	% \psfig in a centering enviornment will behave nicely
	%{\setbox0=\hbox{\ }\ \hskip-\wd0}
	%
	\ps@init@parms
	\parse@ps@parms{#1}
	\compute@sizes
	%
	\ifnum\@p@scost<\@psdraft{
		\if@verbose{
			\typeout{psfig: including \@p@sfile \space }
		}\fi
		%
		\special{ps::[begin] 	\@p@swidth \space \@p@sheight \space
				\@p@sbbllx \space \@p@sbblly \space
				\@p@sbburx \space \@p@sbbury \space
				startTexFig \space }
		\if@clip{
			\if@verbose{
				\typeout{(clip)}
			}\fi
			\special{ps:: doclip \space }
		}\fi
		\if@prologfile
		    \special{ps: plotfile \@prologfileval \space } \fi
		\special{ps: plotfile \@p@sfile \space }
		\if@postlogfile
		    \special{ps: plotfile \@postlogfileval \space } \fi
		\special{ps::[end] endTexFig \space }
		% Create the vbox to reserve the space for the figure
		\vbox to \@p@srheight true sp{
			\hbox to \@p@srwidth true sp{
				\hss
			}
		\vss
		}
	}\else{
		% draft figure, just reserve the space and print the
		% path name.
		\vbox to \@p@srheight true sp{
		\vss
			\hbox to \@p@srwidth true sp{
				\hss
				\if@verbose{
					\@p@sfile
				}\fi
				\hss
			}
		\vss
		}
	}\fi
}}
\def\psglobal{\typeout{psfig: PSGLOBAL is OBSOLETE; use psprint -m instead}}
\psfigRestoreAt







\newcommand{\ldbrack}{[\![}
\newcommand{\rdbrack}{]\!]}
\newcommand{\rdbrackPS}{]\!]_{ps}}
\newcommand{\minus}{{\tt neg}}
\newcommand{\update}{{\tt update}}
\newcommand{\updateSto}{{\tt updateSto}}
\newcommand{\updateA}{{\tt updateA}}
%\newcommand{\rdAddr}{{\tt rdAddr}}
\newcommand{\rdSto}{{\tt rdSto}}
\newcommand{\getSto}{{\tt getSto}}
\newcommand{\getSBRS}{{\tt getSBRS}}
\newcommand{\setSto}{{\tt setSto}}
\newcommand{\rdLabel}{{\tt rdLabel}}
\newcommand{\rdCode}{{\tt rdCode}}
\newcommand{\getCode}{{\tt getCode}}
\newcommand{\callcc}{{\tt callcc}\;}
\newcommand{\call}{{\tt call}}
\newcommand{\acall}{{\tt acall}}
\newcommand{\readIt}{{\tt read}}
\newcommand{\bool}{{\bf bool}}


\newcommand{\getDisplay}{{\tt getDisplay}}
\newcommand{\updateDisplay}{{\tt updateDisplay}}
%\newcommand{\MachLang}{\textsf{\textbf{MachLang}}\/}
\newcommand{\MachLang}{\ensuremath{\mathbf{\mathsf{MachLang}\/}}}
\newcommand{\Rhs}{\ensuremath{\mathbf{\mathsf{Rhs}\/}}}
\newcommand{\TargetLang}{\ensuremath{\mathbf{\mathsf{TargetLang}\/}}}
\newcommand{\IntProducer}{\ensuremath{\mathbf{\mathsf{IntProducer}\/}}}

\newcommand{\letclosed}{{\bf letclosed}}
\newcommand{\letopen}{{\bf letopen}}
\newcommand{\letrec}{{\bf letrec}}
\newcommand{\funcall}{{\bf funcall}}
\newcommand{\apply}{\mathsf{apply}}
\newcommand{\Dynam}{\mathsf{Dynam}}
\newcommand{\Static}{\mathsf{Static}}
\newcommand{\Dmonad}{\mathsf{D}}
\newcommand{\Smonad}{\mathsf{S}}
\newcommand{\DynamOf}[1]{\mathsf{Dynam}(#1)}
\newcommand{\StaticOf}[1]{\mathsf{Static}(#1)}
\newcommand{\compile}{\ensuremath{\mathsf{compile}}}
%\newcommand{\compile}{\textsf{compile}}
\newcommand{\Monad}{\mathsf{M}}
\newcommand{\St}{\mathsf{St}}
\newcommand{\STtwo}{\mathsf{St2}}
\newcommand{\EnvSt}{\mathsf{EnvSt}}
\newcommand{\CPS}{\mathsf{CPS}}
\newcommand{\Id}{\mathsf{Id}}
\newcommand{\T}{\mathsf{T}}
\newcommand{\M}{\mathcal{M}}
\newcommand{\ALLOC}{{\tt ALLOC}}
\newcommand{\NOP}{{\tt NOP}}
\newcommand{\DEALLOC}{{\tt DEALLOC}}
\newcommand{\BRLEQ}{{\tt BRLEQ}}
%\newcommand{\then}{\,\mbox{{\bf{\tt \large{;}}}}\, }
\newcommand{\then}{\,\mbox{{\bf{\tt ;}}}\, }



\newcommand{\Interp}{\mathsf{Dynam}}
\newcommand{\Comp}{\mathsf{Static}}

\newcommand{\updateL}{{\tt updateL}}
\newcommand{\updateLabel}{{\tt updateLabel}}
\newcommand{\getLabel}{{\tt getLabel}}
\newcommand{\getIt}{{\tt get}}
\newcommand{\setIt}{{\tt set}}
\newcommand{\contents}{{\tt read}}
\newcommand{\add}{{\tt add}}
\newcommand{\NN}{N}
\newcommand{\Mach}{[\![ - ]\!]_{Mach}}

%\newcommand{\IntProd}{{\tiny \mathbf{MachLang}\times \mathbf{Rhs}}}
\newcommand{\IntProd}{\mathit{IntProd}}

\newcommand{\getSP}{{\tt getSP}}
\newcommand{\setSBRS}{{\tt setSBRS}}
\newcommand{\rdLoc}{{\tt rdLoc}}

\newcommand{\jump}{{\tt jump}\,}
\newcommand{\JUMP}{{\tt JUMP}}
\newcommand{\SEGM}{{\tt SEGM}}
\newcommand{\SEGMENT}{{\tt SEGMENT}}
\newcommand{\LABEL}{{\tt LABEL}}
\newcommand{\ENDLABEL}{{\tt ENDLABEL}}
\newcommand{\Mono}{{\cal M}\!{\it ono}}

%\newcommand{\UnitType}{\,{\tt void}}
\newcommand{\Void}{{\tt void}}
%\newcommand{\bindI}{\;\;{\bf bindE}\;\;}
\newcommand{\bindD}{\;\star_{D}\;}
\newcommand{\unitD}{{\bf unit}_D}


%\newcommand{\bind}{\;\;{\bf bind}\;\;}
\newcommand{\bind}{\;\star\;}
\newcommand{\bindid}{\;\star_{Id}\;}
\newcommand{\bindul}{\;\;\bigcirc\hspace{-1.25mm}\!\!\!\!\star\;\;\,\;}

\newcommand{\unit}{{\bf unit}}
\newcommand{\CreateTemp}{{\tt CreateTemp}}
\newcommand{\newloc}{{\tt newloc}}
\newcommand{\newLabel}{{\tt newlabel}}
\newcommand{\rdloc}{{\tt rdloc}}
\newcommand{\newAddr}{{\tt newAddr}}
\newcommand{\newSegment}{{\tt newSegment}}
\newcommand{\newSegments}{{\tt newSegments}}
\newcommand{\updateloc}{{\tt updateloc}}
\newcommand{\lift}{{\tt lift}}
\newcommand{\updateC}{{\tt updateC}}
\newcommand{\updateCode}{{\tt updateCode}}
\newcommand{\storeC}{{\tt storeC}}
\newcommand{\store}{{\tt store}}

\newcommand{\StateT}{{\mathcal T}_{\!\mathsf{St}}}
\newcommand{\EnvT}{{\mathcal T}_{\!\mathsf{Env}}}
\newcommand{\ContT}{{\mathcal T}_{\!\mathsf{CPS}}}

%\newcommand{\StateT}{{\mathcal T}_{\! St}}
%\newcommand{\EnvT}{{\mathcal T}_{\! Env}}

\newcommand{\CodeStore}{Code}

\newcommand{\rdAddr}{{\tt rdAddr}}
\newcommand{\inAddr}{{\tt inAddr}}
\newcommand{\Thread}{{\tt Thread}}
\newcommand{\C}{{\cal C}}
%\newcommand{\bindC}{\;{\bf bindC}\;}
\newcommand{\bindS}{\;\star_{S}\;}
\newcommand{\unitS}{{\bf unit}_S}
\newcommand{\BINDS}[2]{#1\bindS \lambda #2.}
\newcommand{\BIND}[2]{#1\bind \lambda #2.}
\newcommand{\BINDD}[2]{#1\bindD \lambda #2.}
\newcommand{\BINDM}[2]{#1\bindM \lambda #2.}
\newcommand{\MS}[1]{\ensuremath{\M\ldbrack #1 \rdbrack}}
\newcommand{\CS}[1]{\ensuremath{\C\ldbrack #1 \rdbrack}}
\newcommand{\standard}[1]{\ensuremath{\ldbrack #1 \rdbrack}}

\newcommand{\rdSeg}{{\tt rdSeg}}
\newcommand{\rdSD}{{\tt rdSD}}
\newcommand{\inSD}{{\tt inSD}}

\newcommand{\orLog}{\,{\bf or}\,}
\newcommand{\notLog}{{\bf not}\,}
\newcommand{\true}{{\tt true}}
\newcommand{\false}{{\tt false}}
\newcommand{\currentFrame}{{\tt currentFrame}}
\newcommand{\mkCall}{{\tt mkCall}}
\newcommand{\return}{{\tt return}}
\newcommand{\intexp}{{\bf intexp}}
\newcommand{\boolexp}{{\bf boolexp}}
\newcommand{\comm}{{\bf comm}}
%\newcommand{\prop}{{\bf prop}}
\newcommand{\mkcall}{{\tt mkcall}}
\newcommand{\mkargcall}{{\tt mkargcall}}
\newcommand{\mksubr}{{\tt mksubr}}
\newcommand{\initCS}{{\tt init}_{CS}}
\newcommand{\lookup}{{\tt lookup}}
\newcommand{\Assign}{\mathsf{Assign}}
\newcommand{\Negate}{\ensuremath{\mathsf{Negate}}}
\newcommand{\Add}{\mathsf{Add}}
\newcommand{\Arrow}{\mathsf{Lambda}}
\newcommand{\IfThen}{\ensuremath{\mathsf{IfThen}}}
\newcommand{\IfThenPS}{\ensuremath{\mathsf{IfThenPS}}}
\newcommand{\IfThenElsePS}{\ensuremath{\mathsf{IfThenElsePS}}}
\newcommand{\IfThenOf}[4]{\mathsf{IfThen}(#1,#2,#3,#4)}
\newcommand{\WhilePS}{\mathsf{WhilePS}}
\newcommand{\Or}{\mathsf{Or}}
\newcommand{\Not}{\mathsf{Not}}
\newcommand{\Lteq}{\mathsf{Lteq}}
%\newcommand{\Fresh}{\mathsf{Fresh}}
\newcommand{\New}{\mathsf{New}}
\newcommand{\Observe}{\mathsf{Obs}}
%\newcommand{\If}{\mathsf{If}}
\newcommand{\If}[2]{#1 \;\mbox{\raisebox{-0.4ex}{{\huge $\triangleright$}}}\; #2}
\newcommand{\guard}{\;\mbox{\raisebox{-0.4ex}{{\huge $\triangleright$}}}\;}
\newcommand{\NotEqual}{\mathsf{NotEqual}}
\newcommand{\And}{\mathsf{And}}
\newcommand{\NotInRHS}{\mathsf{NotInRHS}}
\newcommand{\DisplayUnchanged}{\mathsf{DisplayUnchanged}}
\newcommand{\savereg}{\mathsf{savereg}}
\newcommand{\rhs}{\mathit{rhs}}
\newcommand{\tmps}{\mathit{tmps}}
\newcommand{\pop}{\mathsf{pop}}

\newcommand{\Alloc}{{\tt Alloc}}
\newcommand{\deAlloc}{{\tt deAlloc}}

\newcommand{\rdEnv}{{\tt rdEnv}}
\newcommand{\rdgen}{{\tt rd}}
\newcommand{\ingen}{{\tt in}}
\newcommand{\inEnv}{{\tt inEnv}}
\newcommand{\allocLoc}{{\tt allocLoc}}

\newlength{\bracket}
\settowidth{\bracket}{[}
\newsavebox{\leftdoublebracket}
\newsavebox{\rightdoublebracket}
\sbox{\leftdoublebracket}{[\hspace{-.5\bracket}[}
\sbox{\rightdoublebracket}{]\hspace{-.5\bracket}]}
\newcommand{\ldbra}{\usebox{\leftdoublebracket}}
\newcommand{\rdbra}{\usebox{\rightdoublebracket}}
\newsavebox{\leftbraced}
\newsavebox{\rightbraced} 
\sbox{\leftbraced}{\{\hspace{-.5\bracket}|}
\sbox{\rightbraced}{|\hspace{-.5\bracket}\}}
\newcommand{\lbrac}{\usebox{\leftbraced}}
\newcommand{\rbrac}{\usebox{\rightbraced}}
\makeatother
%% Psfig/TeX Release 1.3
% dvips version
%
% All software, documentation, and related files in this distribution of
% psfig/tex are Copyright 1987, 1988 Trevor J. Darrell
%
% Permission is granted for use and non-profit distribution of psfig/tex 
% providing that this notice be clearly maintained, but the right to
% distribute any portion of psfig/tex for profit or as part of any commercial
% product is specifically reserved for the author.
%
% $Header: /cvs/programatica/tools/interp/HaskellWorkshop/psfig.tex,v 1.1 2001/04/25 23:52:51 wlh Exp $
% $Source: /cvs/programatica/tools/interp/HaskellWorkshop/psfig.tex,v $
%
% Thanks to Greg Hager (GDH) and Ned Batchelder for their contributions
% to this project.
%
% Modified by J. Daniel Smith on 9 October 1990 to accept the
% %%BoundingBox: comment with or without a space after the colon.  Stole
% file reading code from Tom Rokicki's EPSF.TEX file (see below).

%
% from a suggestion by eijkhout@csrd.uiuc.edu to allow
% loading as a style file:
\edef\psfigRestoreAt{\catcode`@=\number\catcode`@\relax}
\catcode`\@=11\relax
\newwrite\@unused
\def\typeout#1{{\let\protect\string\immediate\write\@unused{#1}}}
\typeout{psfig/tex 1.4-dvips}


%% Here's how you define your figure path.  Should be set up with null
%% default and a user useable definition.

\def\figurepath{./}
\def\psfigurepath#1{\edef\figurepath{#1}}

%
% @psdo control structure -- similar to Latex @for.
% I redefined these with different names so that psfig can
% be used with TeX as well as LaTeX, and so that it will not 
% be vunerable to future changes in LaTeX's internal
% control structure,
%
\def\@nnil{\@nil}
\def\@empty{}
\def\@psdonoop#1\@@#2#3{}
\def\@psdo#1:=#2\do#3{\edef\@psdotmp{#2}\ifx\@psdotmp\@empty \else
    \expandafter\@psdoloop#2,\@nil,\@nil\@@#1{#3}\fi}
\def\@psdoloop#1,#2,#3\@@#4#5{\def#4{#1}\ifx #4\@nnil \else
       #5\def#4{#2}\ifx #4\@nnil \else#5\@ipsdoloop #3\@@#4{#5}\fi\fi}
\def\@ipsdoloop#1,#2\@@#3#4{\def#3{#1}\ifx #3\@nnil 
       \let\@nextwhile=\@psdonoop \else
      #4\relax\let\@nextwhile=\@ipsdoloop\fi\@nextwhile#2\@@#3{#4}}
\def\@tpsdo#1:=#2\do#3{\xdef\@psdotmp{#2}\ifx\@psdotmp\@empty \else
    \@tpsdoloop#2\@nil\@nil\@@#1{#3}\fi}
\def\@tpsdoloop#1#2\@@#3#4{\def#3{#1}\ifx #3\@nnil 
       \let\@nextwhile=\@psdonoop \else
      #4\relax\let\@nextwhile=\@tpsdoloop\fi\@nextwhile#2\@@#3{#4}}
% 
%
%%%%%%%%%%%%%%%%%%%%%%%%%%%%%%%%%%%%%%%%%%%%%%%%%%%%%%%%%%%%%%%%%%%
% file reading stuff from epsf.tex
%   EPSF.TEX macro file:
%   Written by Tomas Rokicki of Radical Eye Software, 29 Mar 1989.
%   Revised by Don Knuth, 3 Jan 1990.
%   Revised by Tomas Rokicki to accept bounding boxes with no
%      space after the colon, 18 Jul 1990.
%   Portions modified/removed for use in PSFIG package by
%      J. Daniel Smith, 9 October 1990.
%
\newread\ps@stream
\newif\ifnot@eof       % continue looking for the bounding box?
\newif\if@noisy        % report what you're making?
\newif\if@atend        % %%BoundingBox: has (at end) specification
\newif\if@psfile       % does this look like a PostScript file?
%
% PostScript files should start with `%!'
%
{\catcode`\%=12\global\gdef\epsf@start{%!}}
\def\epsf@PS{PS}
%
\def\epsf@getbb#1{%
%
%   The first thing we need to do is to open the
%   PostScript file, if possible.
%
\openin\ps@stream=#1
\ifeof\ps@stream\typeout{Error, File #1 not found}\else
%
%   Okay, we got it. Now we'll scan lines until we find one that doesn't
%   start with %. We're looking for the bounding box comment.
%
   {\not@eoftrue \chardef\other=12
    \def\do##1{\catcode`##1=\other}\dospecials \catcode`\ =10
    \loop
       \if@psfile
	  \read\ps@stream to \epsf@fileline
       \else{
	  \obeyspaces
          \read\ps@stream to \epsf@tmp\global\let\epsf@fileline\epsf@tmp}
       \fi
       \ifeof\ps@stream\not@eoffalse\else
%
%   Check the first line for `%!'.  Issue a warning message if its not
%   there, since the file might not be a PostScript file.
%
       \if@psfile\else
       \expandafter\epsf@test\epsf@fileline:. \\%
       \fi
%
%   We check to see if the first character is a % sign;
%   if so, we look further and stop only if the line begins with
%   `%%BoundingBox:' and the `(atend)' specification was not found.
%   That is, the only way to stop is when the end of file is reached,
%   or a `%%BoundingBox: llx lly urx ury' line is found.
%
          \expandafter\epsf@aux\epsf@fileline:. \\%
       \fi
   \ifnot@eof\repeat
   }\closein\ps@stream\fi}%
%
% This tests if the file we are reading looks like a PostScript file.
%
\long\def\epsf@test#1#2#3:#4\\{\def\epsf@testit{#1#2}
			\ifx\epsf@testit\epsf@start\else
\typeout{Warning! File does not start with `\epsf@start'.  It may not be a PostScript file.}
			\fi
			\@psfiletrue} % don't test after 1st line
%
%   We still need to define the tricky \epsf@aux macro. This requires
%   a couple of magic constants for comparison purposes.
%
{\catcode`\%=12\global\let\epsf@percent=%\global\def\epsf@bblit{%BoundingBox}}
%
%
%   So we're ready to check for `%BoundingBox:' and to grab the
%   values if they are found.  We continue searching if `(at end)'
%   was found after the `%BoundingBox:'.
%
\long\def\epsf@aux#1#2:#3\\{\ifx#1\epsf@percent
   \def\epsf@testit{#2}\ifx\epsf@testit\epsf@bblit
	\@atendfalse
        \epsf@atend #3 . \\%
	\if@atend	
	   \if@verbose{
		\typeout{psfig: found `(atend)'; continuing search}
	   }\fi
        \else
        \epsf@grab #3 . . . \\%
        \not@eoffalse
        \global\no@bbfalse
        \fi
   \fi\fi}%
%
%   Here we grab the values and stuff them in the appropriate definitions.
%
\def\epsf@grab #1 #2 #3 #4 #5\\{%
   \global\def\epsf@llx{#1}\ifx\epsf@llx\empty
      \epsf@grab #2 #3 #4 #5 .\\\else
   \global\def\epsf@lly{#2}%
   \global\def\epsf@urx{#3}\global\def\epsf@ury{#4}\fi}%
%
% Determine if the stuff following the %%BoundingBox is `(atend)'
% J. Daniel Smith.  Copied from \epsf@grab above.
%
\def\epsf@atendlit{(atend)} 
\def\epsf@atend #1 #2 #3\\{%
   \def\epsf@tmp{#1}\ifx\epsf@tmp\empty
      \epsf@atend #2 #3 .\\\else
   \ifx\epsf@tmp\epsf@atendlit\@atendtrue\fi\fi}


% End of file reading stuff from epsf.tex
%%%%%%%%%%%%%%%%%%%%%%%%%%%%%%%%%%%%%%%%%%%%%%%%%%%%%%%%%%%%%%%%%%%

\def\psdraft{
	\def\@psdraft{0}
	%\typeout{draft level now is \@psdraft \space . }
}
\def\psfull{
	\def\@psdraft{100}
	%\typeout{draft level now is \@psdraft \space . }
}
\psfull
\newif\if@prologfile
\newif\if@postlogfile
\def\pssilent{
	\@noisyfalse
}
\def\psnoisy{
	\@noisytrue
}
\psnoisy
%%% These are for the option list.
%%% A specification of the form a = b maps to calling \@p@@sa{b}
\newif\if@bbllx
\newif\if@bblly
\newif\if@bburx
\newif\if@bbury
\newif\if@height
\newif\if@width
\newif\if@rheight
\newif\if@rwidth
\newif\if@clip
\newif\if@verbose
\def\@p@@sclip#1{\@cliptrue}

%%% GDH 7/26/87 -- changed so that it first looks in the local directory,
%%% then in a specified global directory for the ps file.

\def\@p@@sfile#1{\def\@p@sfile{null}%
	        \openin1=#1
		\ifeof1\closein1%
		       \openin1=\figurepath#1
			\ifeof1\typeout{Error, File #1 not found}
			\else\closein1
			    \edef\@p@sfile{\figurepath#1}%
                        \fi%
		 \else\closein1%
		       \def\@p@sfile{#1}%
		 \fi}
\def\@p@@sfigure#1{\def\@p@sfile{null}%
	        \openin1=#1
		\ifeof1\closein1%
		       \openin1=\figurepath#1
			\ifeof1\typeout{Error, File #1 not found}
			\else\closein1
			    \def\@p@sfile{\figurepath#1}%
                        \fi%
		 \else\closein1%
		       \def\@p@sfile{#1}%
		 \fi}

\def\@p@@sbbllx#1{
		%\typeout{bbllx is #1}
		\@bbllxtrue
		\dimen100=#1
		\edef\@p@sbbllx{\number\dimen100}
}
\def\@p@@sbblly#1{
		%\typeout{bblly is #1}
		\@bbllytrue
		\dimen100=#1
		\edef\@p@sbblly{\number\dimen100}
}
\def\@p@@sbburx#1{
		%\typeout{bburx is #1}
		\@bburxtrue
		\dimen100=#1
		\edef\@p@sbburx{\number\dimen100}
}
\def\@p@@sbbury#1{
		%\typeout{bbury is #1}
		\@bburytrue
		\dimen100=#1
		\edef\@p@sbbury{\number\dimen100}
}
\def\@p@@sheight#1{
		\@heighttrue
		\dimen100=#1
   		\edef\@p@sheight{\number\dimen100}
		%\typeout{Height is \@p@sheight}
}
\def\@p@@swidth#1{
		%\typeout{Width is #1}
		\@widthtrue
		\dimen100=#1
		\edef\@p@swidth{\number\dimen100}
}
\def\@p@@srheight#1{
		%\typeout{Reserved height is #1}
		\@rheighttrue
		\dimen100=#1
		\edef\@p@srheight{\number\dimen100}
}
\def\@p@@srwidth#1{
		%\typeout{Reserved width is #1}
		\@rwidthtrue
		\dimen100=#1
		\edef\@p@srwidth{\number\dimen100}
}
\def\@p@@ssilent#1{ 
		\@verbosefalse
}
\def\@p@@sprolog#1{\@prologfiletrue\def\@prologfileval{#1}}
\def\@p@@spostlog#1{\@postlogfiletrue\def\@postlogfileval{#1}}
\def\@cs@name#1{\csname #1\endcsname}
\def\@setparms#1=#2,{\@cs@name{@p@@s#1}{#2}}
%
% initialize the defaults (size the size of the figure)
%
\def\ps@init@parms{
		\@bbllxfalse \@bbllyfalse
		\@bburxfalse \@bburyfalse
		\@heightfalse \@widthfalse
		\@rheightfalse \@rwidthfalse
		\def\@p@sbbllx{}\def\@p@sbblly{}
		\def\@p@sbburx{}\def\@p@sbbury{}
		\def\@p@sheight{}\def\@p@swidth{}
		\def\@p@srheight{}\def\@p@srwidth{}
		\def\@p@sfile{}
		\def\@p@scost{10}
		\def\@sc{}
		\@prologfilefalse
		\@postlogfilefalse
		\@clipfalse
		\if@noisy
			\@verbosetrue
		\else
			\@verbosefalse
		\fi
}
%
% Go through the options setting things up.
%
\def\parse@ps@parms#1{
	 	\@psdo\@psfiga:=#1\do
		   {\expandafter\@setparms\@psfiga,}}
%
% Compute bb height and width
%
\newif\ifno@bb
\def\bb@missing{
	\if@verbose{
		\typeout{psfig: searching \@p@sfile \space  for bounding box}
	}\fi
	\no@bbtrue
	\epsf@getbb{\@p@sfile}
        \ifno@bb \else \bb@cull\epsf@llx\epsf@lly\epsf@urx\epsf@ury\fi
}	
\def\bb@cull#1#2#3#4{
	\dimen100=#1 bp\edef\@p@sbbllx{\number\dimen100}
	\dimen100=#2 bp\edef\@p@sbblly{\number\dimen100}
	\dimen100=#3 bp\edef\@p@sbburx{\number\dimen100}
	\dimen100=#4 bp\edef\@p@sbbury{\number\dimen100}
	\no@bbfalse
}
\def\compute@bb{
		\no@bbfalse
		\if@bbllx \else \no@bbtrue \fi
		\if@bblly \else \no@bbtrue \fi
		\if@bburx \else \no@bbtrue \fi
		\if@bbury \else \no@bbtrue \fi
		\ifno@bb \bb@missing \fi
		\ifno@bb \typeout{FATAL ERROR: no bb supplied or found}
			\no-bb-error
		\fi
		%
		\count203=\@p@sbburx
		\count204=\@p@sbbury
		\advance\count203 by -\@p@sbbllx
		\advance\count204 by -\@p@sbblly
		\edef\@bbw{\number\count203}
		\edef\@bbh{\number\count204}
		%\typeout{ bbh = \@bbh, bbw = \@bbw }
}
%
% \in@hundreds performs #1 * (#2 / #3) correct to the hundreds,
%	then leaves the result in @result
%
\def\in@hundreds#1#2#3{\count240=#2 \count241=#3
		     \count100=\count240	% 100 is first digit #2/#3
		     \divide\count100 by \count241
		     \count101=\count100
		     \multiply\count101 by \count241
		     \advance\count240 by -\count101
		     \multiply\count240 by 10
		     \count101=\count240	%101 is second digit of #2/#3
		     \divide\count101 by \count241
		     \count102=\count101
		     \multiply\count102 by \count241
		     \advance\count240 by -\count102
		     \multiply\count240 by 10
		     \count102=\count240	% 102 is the third digit
		     \divide\count102 by \count241
		     \count200=#1\count205=0
		     \count201=\count200
			\multiply\count201 by \count100
		 	\advance\count205 by \count201
		     \count201=\count200
			\divide\count201 by 10
			\multiply\count201 by \count101
			\advance\count205 by \count201
			%
		     \count201=\count200
			\divide\count201 by 100
			\multiply\count201 by \count102
			\advance\count205 by \count201
			%
		     \edef\@result{\number\count205}
}
\def\compute@wfromh{
		% computing : width = height * (bbw / bbh)
		\in@hundreds{\@p@sheight}{\@bbw}{\@bbh}
		%\typeout{ \@p@sheight * \@bbw / \@bbh, = \@result }
		\edef\@p@swidth{\@result}
		%\typeout{w from h: width is \@p@swidth}
}
\def\compute@hfromw{
		% computing : height = width * (bbh / bbw)
		\in@hundreds{\@p@swidth}{\@bbh}{\@bbw}
		%\typeout{ \@p@swidth * \@bbh / \@bbw = \@result }
		\edef\@p@sheight{\@result}
		%\typeout{h from w : height is \@p@sheight}
}
\def\compute@handw{
		\if@height 
			\if@width
			\else
				\compute@wfromh
			\fi
		\else 
			\if@width
				\compute@hfromw
			\else
				\edef\@p@sheight{\@bbh}
				\edef\@p@swidth{\@bbw}
			\fi
		\fi
}
\def\compute@resv{
		\if@rheight \else \edef\@p@srheight{\@p@sheight} \fi
		\if@rwidth \else \edef\@p@srwidth{\@p@swidth} \fi
}
%		
% Compute any missing values
\def\compute@sizes{
	\compute@bb
	\compute@handw
	\compute@resv
}
%
% \psfig
% usage : \psfig{file=, height=, width=, bbllx=, bblly=, bburx=, bbury=,
%			rheight=, rwidth=, clip=}
%
% "clip=" is a switch and takes no value, but the `=' must be present.
\def\psfig#1{\vbox {
	% do a zero width hard space so that a single
	% \psfig in a centering enviornment will behave nicely
	%{\setbox0=\hbox{\ }\ \hskip-\wd0}
	%
	\ps@init@parms
	\parse@ps@parms{#1}
	\compute@sizes
	%
	\ifnum\@p@scost<\@psdraft{
		\if@verbose{
			\typeout{psfig: including \@p@sfile \space }
		}\fi
		%
		\special{ps::[begin] 	\@p@swidth \space \@p@sheight \space
				\@p@sbbllx \space \@p@sbblly \space
				\@p@sbburx \space \@p@sbbury \space
				startTexFig \space }
		\if@clip{
			\if@verbose{
				\typeout{(clip)}
			}\fi
			\special{ps:: doclip \space }
		}\fi
		\if@prologfile
		    \special{ps: plotfile \@prologfileval \space } \fi
		\special{ps: plotfile \@p@sfile \space }
		\if@postlogfile
		    \special{ps: plotfile \@postlogfileval \space } \fi
		\special{ps::[end] endTexFig \space }
		% Create the vbox to reserve the space for the figure
		\vbox to \@p@srheight true sp{
			\hbox to \@p@srwidth true sp{
				\hss
			}
		\vss
		}
	}\else{
		% draft figure, just reserve the space and print the
		% path name.
		\vbox to \@p@srheight true sp{
		\vss
			\hbox to \@p@srwidth true sp{
				\hss
				\if@verbose{
					\@p@sfile
				}\fi
				\hss
			}
		\vss
		}
	}\fi
}}
\def\psglobal{\typeout{psfig: PSGLOBAL is OBSOLETE; use psprint -m instead}}
\psfigRestoreAt






				% <---- load psfig macros
\newcommand{\psinput}[1]{
  \mbox{\psfig{figure=#1}}}
\newcommand{\hgap}{\hspace{12pt}}
\newcommand{\vgap}{\\[12pt]}
\newcommand{\pscap}[2]{
  \parbox{2.01in}{
   \begin{center}
    \psinput{#1} \\ #2
   \end{center}
  }
}

%% Psfig/TeX Release 1.3
% dvips version
%
% All software, documentation, and related files in this distribution of
% psfig/tex are Copyright 1987, 1988 Trevor J. Darrell
%
% Permission is granted for use and non-profit distribution of psfig/tex 
% providing that this notice be clearly maintained, but the right to
% distribute any portion of psfig/tex for profit or as part of any commercial
% product is specifically reserved for the author.
%
% $Header: /cvs/programatica/tools/interp/HaskellWorkshop/psfig.tex,v 1.1 2001/04/25 23:52:51 wlh Exp $
% $Source: /cvs/programatica/tools/interp/HaskellWorkshop/psfig.tex,v $
%
% Thanks to Greg Hager (GDH) and Ned Batchelder for their contributions
% to this project.
%
% Modified by J. Daniel Smith on 9 October 1990 to accept the
% %%BoundingBox: comment with or without a space after the colon.  Stole
% file reading code from Tom Rokicki's EPSF.TEX file (see below).

%
% from a suggestion by eijkhout@csrd.uiuc.edu to allow
% loading as a style file:
\edef\psfigRestoreAt{\catcode`@=\number\catcode`@\relax}
\catcode`\@=11\relax
\newwrite\@unused
\def\typeout#1{{\let\protect\string\immediate\write\@unused{#1}}}
\typeout{psfig/tex 1.4-dvips}


%% Here's how you define your figure path.  Should be set up with null
%% default and a user useable definition.

\def\figurepath{./}
\def\psfigurepath#1{\edef\figurepath{#1}}

%
% @psdo control structure -- similar to Latex @for.
% I redefined these with different names so that psfig can
% be used with TeX as well as LaTeX, and so that it will not 
% be vunerable to future changes in LaTeX's internal
% control structure,
%
\def\@nnil{\@nil}
\def\@empty{}
\def\@psdonoop#1\@@#2#3{}
\def\@psdo#1:=#2\do#3{\edef\@psdotmp{#2}\ifx\@psdotmp\@empty \else
    \expandafter\@psdoloop#2,\@nil,\@nil\@@#1{#3}\fi}
\def\@psdoloop#1,#2,#3\@@#4#5{\def#4{#1}\ifx #4\@nnil \else
       #5\def#4{#2}\ifx #4\@nnil \else#5\@ipsdoloop #3\@@#4{#5}\fi\fi}
\def\@ipsdoloop#1,#2\@@#3#4{\def#3{#1}\ifx #3\@nnil 
       \let\@nextwhile=\@psdonoop \else
      #4\relax\let\@nextwhile=\@ipsdoloop\fi\@nextwhile#2\@@#3{#4}}
\def\@tpsdo#1:=#2\do#3{\xdef\@psdotmp{#2}\ifx\@psdotmp\@empty \else
    \@tpsdoloop#2\@nil\@nil\@@#1{#3}\fi}
\def\@tpsdoloop#1#2\@@#3#4{\def#3{#1}\ifx #3\@nnil 
       \let\@nextwhile=\@psdonoop \else
      #4\relax\let\@nextwhile=\@tpsdoloop\fi\@nextwhile#2\@@#3{#4}}
% 
%
%%%%%%%%%%%%%%%%%%%%%%%%%%%%%%%%%%%%%%%%%%%%%%%%%%%%%%%%%%%%%%%%%%%
% file reading stuff from epsf.tex
%   EPSF.TEX macro file:
%   Written by Tomas Rokicki of Radical Eye Software, 29 Mar 1989.
%   Revised by Don Knuth, 3 Jan 1990.
%   Revised by Tomas Rokicki to accept bounding boxes with no
%      space after the colon, 18 Jul 1990.
%   Portions modified/removed for use in PSFIG package by
%      J. Daniel Smith, 9 October 1990.
%
\newread\ps@stream
\newif\ifnot@eof       % continue looking for the bounding box?
\newif\if@noisy        % report what you're making?
\newif\if@atend        % %%BoundingBox: has (at end) specification
\newif\if@psfile       % does this look like a PostScript file?
%
% PostScript files should start with `%!'
%
{\catcode`\%=12\global\gdef\epsf@start{%!}}
\def\epsf@PS{PS}
%
\def\epsf@getbb#1{%
%
%   The first thing we need to do is to open the
%   PostScript file, if possible.
%
\openin\ps@stream=#1
\ifeof\ps@stream\typeout{Error, File #1 not found}\else
%
%   Okay, we got it. Now we'll scan lines until we find one that doesn't
%   start with %. We're looking for the bounding box comment.
%
   {\not@eoftrue \chardef\other=12
    \def\do##1{\catcode`##1=\other}\dospecials \catcode`\ =10
    \loop
       \if@psfile
	  \read\ps@stream to \epsf@fileline
       \else{
	  \obeyspaces
          \read\ps@stream to \epsf@tmp\global\let\epsf@fileline\epsf@tmp}
       \fi
       \ifeof\ps@stream\not@eoffalse\else
%
%   Check the first line for `%!'.  Issue a warning message if its not
%   there, since the file might not be a PostScript file.
%
       \if@psfile\else
       \expandafter\epsf@test\epsf@fileline:. \\%
       \fi
%
%   We check to see if the first character is a % sign;
%   if so, we look further and stop only if the line begins with
%   `%%BoundingBox:' and the `(atend)' specification was not found.
%   That is, the only way to stop is when the end of file is reached,
%   or a `%%BoundingBox: llx lly urx ury' line is found.
%
          \expandafter\epsf@aux\epsf@fileline:. \\%
       \fi
   \ifnot@eof\repeat
   }\closein\ps@stream\fi}%
%
% This tests if the file we are reading looks like a PostScript file.
%
\long\def\epsf@test#1#2#3:#4\\{\def\epsf@testit{#1#2}
			\ifx\epsf@testit\epsf@start\else
\typeout{Warning! File does not start with `\epsf@start'.  It may not be a PostScript file.}
			\fi
			\@psfiletrue} % don't test after 1st line
%
%   We still need to define the tricky \epsf@aux macro. This requires
%   a couple of magic constants for comparison purposes.
%
{\catcode`\%=12\global\let\epsf@percent=%\global\def\epsf@bblit{%BoundingBox}}
%
%
%   So we're ready to check for `%BoundingBox:' and to grab the
%   values if they are found.  We continue searching if `(at end)'
%   was found after the `%BoundingBox:'.
%
\long\def\epsf@aux#1#2:#3\\{\ifx#1\epsf@percent
   \def\epsf@testit{#2}\ifx\epsf@testit\epsf@bblit
	\@atendfalse
        \epsf@atend #3 . \\%
	\if@atend	
	   \if@verbose{
		\typeout{psfig: found `(atend)'; continuing search}
	   }\fi
        \else
        \epsf@grab #3 . . . \\%
        \not@eoffalse
        \global\no@bbfalse
        \fi
   \fi\fi}%
%
%   Here we grab the values and stuff them in the appropriate definitions.
%
\def\epsf@grab #1 #2 #3 #4 #5\\{%
   \global\def\epsf@llx{#1}\ifx\epsf@llx\empty
      \epsf@grab #2 #3 #4 #5 .\\\else
   \global\def\epsf@lly{#2}%
   \global\def\epsf@urx{#3}\global\def\epsf@ury{#4}\fi}%
%
% Determine if the stuff following the %%BoundingBox is `(atend)'
% J. Daniel Smith.  Copied from \epsf@grab above.
%
\def\epsf@atendlit{(atend)} 
\def\epsf@atend #1 #2 #3\\{%
   \def\epsf@tmp{#1}\ifx\epsf@tmp\empty
      \epsf@atend #2 #3 .\\\else
   \ifx\epsf@tmp\epsf@atendlit\@atendtrue\fi\fi}


% End of file reading stuff from epsf.tex
%%%%%%%%%%%%%%%%%%%%%%%%%%%%%%%%%%%%%%%%%%%%%%%%%%%%%%%%%%%%%%%%%%%

\def\psdraft{
	\def\@psdraft{0}
	%\typeout{draft level now is \@psdraft \space . }
}
\def\psfull{
	\def\@psdraft{100}
	%\typeout{draft level now is \@psdraft \space . }
}
\psfull
\newif\if@prologfile
\newif\if@postlogfile
\def\pssilent{
	\@noisyfalse
}
\def\psnoisy{
	\@noisytrue
}
\psnoisy
%%% These are for the option list.
%%% A specification of the form a = b maps to calling \@p@@sa{b}
\newif\if@bbllx
\newif\if@bblly
\newif\if@bburx
\newif\if@bbury
\newif\if@height
\newif\if@width
\newif\if@rheight
\newif\if@rwidth
\newif\if@clip
\newif\if@verbose
\def\@p@@sclip#1{\@cliptrue}

%%% GDH 7/26/87 -- changed so that it first looks in the local directory,
%%% then in a specified global directory for the ps file.

\def\@p@@sfile#1{\def\@p@sfile{null}%
	        \openin1=#1
		\ifeof1\closein1%
		       \openin1=\figurepath#1
			\ifeof1\typeout{Error, File #1 not found}
			\else\closein1
			    \edef\@p@sfile{\figurepath#1}%
                        \fi%
		 \else\closein1%
		       \def\@p@sfile{#1}%
		 \fi}
\def\@p@@sfigure#1{\def\@p@sfile{null}%
	        \openin1=#1
		\ifeof1\closein1%
		       \openin1=\figurepath#1
			\ifeof1\typeout{Error, File #1 not found}
			\else\closein1
			    \def\@p@sfile{\figurepath#1}%
                        \fi%
		 \else\closein1%
		       \def\@p@sfile{#1}%
		 \fi}

\def\@p@@sbbllx#1{
		%\typeout{bbllx is #1}
		\@bbllxtrue
		\dimen100=#1
		\edef\@p@sbbllx{\number\dimen100}
}
\def\@p@@sbblly#1{
		%\typeout{bblly is #1}
		\@bbllytrue
		\dimen100=#1
		\edef\@p@sbblly{\number\dimen100}
}
\def\@p@@sbburx#1{
		%\typeout{bburx is #1}
		\@bburxtrue
		\dimen100=#1
		\edef\@p@sbburx{\number\dimen100}
}
\def\@p@@sbbury#1{
		%\typeout{bbury is #1}
		\@bburytrue
		\dimen100=#1
		\edef\@p@sbbury{\number\dimen100}
}
\def\@p@@sheight#1{
		\@heighttrue
		\dimen100=#1
   		\edef\@p@sheight{\number\dimen100}
		%\typeout{Height is \@p@sheight}
}
\def\@p@@swidth#1{
		%\typeout{Width is #1}
		\@widthtrue
		\dimen100=#1
		\edef\@p@swidth{\number\dimen100}
}
\def\@p@@srheight#1{
		%\typeout{Reserved height is #1}
		\@rheighttrue
		\dimen100=#1
		\edef\@p@srheight{\number\dimen100}
}
\def\@p@@srwidth#1{
		%\typeout{Reserved width is #1}
		\@rwidthtrue
		\dimen100=#1
		\edef\@p@srwidth{\number\dimen100}
}
\def\@p@@ssilent#1{ 
		\@verbosefalse
}
\def\@p@@sprolog#1{\@prologfiletrue\def\@prologfileval{#1}}
\def\@p@@spostlog#1{\@postlogfiletrue\def\@postlogfileval{#1}}
\def\@cs@name#1{\csname #1\endcsname}
\def\@setparms#1=#2,{\@cs@name{@p@@s#1}{#2}}
%
% initialize the defaults (size the size of the figure)
%
\def\ps@init@parms{
		\@bbllxfalse \@bbllyfalse
		\@bburxfalse \@bburyfalse
		\@heightfalse \@widthfalse
		\@rheightfalse \@rwidthfalse
		\def\@p@sbbllx{}\def\@p@sbblly{}
		\def\@p@sbburx{}\def\@p@sbbury{}
		\def\@p@sheight{}\def\@p@swidth{}
		\def\@p@srheight{}\def\@p@srwidth{}
		\def\@p@sfile{}
		\def\@p@scost{10}
		\def\@sc{}
		\@prologfilefalse
		\@postlogfilefalse
		\@clipfalse
		\if@noisy
			\@verbosetrue
		\else
			\@verbosefalse
		\fi
}
%
% Go through the options setting things up.
%
\def\parse@ps@parms#1{
	 	\@psdo\@psfiga:=#1\do
		   {\expandafter\@setparms\@psfiga,}}
%
% Compute bb height and width
%
\newif\ifno@bb
\def\bb@missing{
	\if@verbose{
		\typeout{psfig: searching \@p@sfile \space  for bounding box}
	}\fi
	\no@bbtrue
	\epsf@getbb{\@p@sfile}
        \ifno@bb \else \bb@cull\epsf@llx\epsf@lly\epsf@urx\epsf@ury\fi
}	
\def\bb@cull#1#2#3#4{
	\dimen100=#1 bp\edef\@p@sbbllx{\number\dimen100}
	\dimen100=#2 bp\edef\@p@sbblly{\number\dimen100}
	\dimen100=#3 bp\edef\@p@sbburx{\number\dimen100}
	\dimen100=#4 bp\edef\@p@sbbury{\number\dimen100}
	\no@bbfalse
}
\def\compute@bb{
		\no@bbfalse
		\if@bbllx \else \no@bbtrue \fi
		\if@bblly \else \no@bbtrue \fi
		\if@bburx \else \no@bbtrue \fi
		\if@bbury \else \no@bbtrue \fi
		\ifno@bb \bb@missing \fi
		\ifno@bb \typeout{FATAL ERROR: no bb supplied or found}
			\no-bb-error
		\fi
		%
		\count203=\@p@sbburx
		\count204=\@p@sbbury
		\advance\count203 by -\@p@sbbllx
		\advance\count204 by -\@p@sbblly
		\edef\@bbw{\number\count203}
		\edef\@bbh{\number\count204}
		%\typeout{ bbh = \@bbh, bbw = \@bbw }
}
%
% \in@hundreds performs #1 * (#2 / #3) correct to the hundreds,
%	then leaves the result in @result
%
\def\in@hundreds#1#2#3{\count240=#2 \count241=#3
		     \count100=\count240	% 100 is first digit #2/#3
		     \divide\count100 by \count241
		     \count101=\count100
		     \multiply\count101 by \count241
		     \advance\count240 by -\count101
		     \multiply\count240 by 10
		     \count101=\count240	%101 is second digit of #2/#3
		     \divide\count101 by \count241
		     \count102=\count101
		     \multiply\count102 by \count241
		     \advance\count240 by -\count102
		     \multiply\count240 by 10
		     \count102=\count240	% 102 is the third digit
		     \divide\count102 by \count241
		     \count200=#1\count205=0
		     \count201=\count200
			\multiply\count201 by \count100
		 	\advance\count205 by \count201
		     \count201=\count200
			\divide\count201 by 10
			\multiply\count201 by \count101
			\advance\count205 by \count201
			%
		     \count201=\count200
			\divide\count201 by 100
			\multiply\count201 by \count102
			\advance\count205 by \count201
			%
		     \edef\@result{\number\count205}
}
\def\compute@wfromh{
		% computing : width = height * (bbw / bbh)
		\in@hundreds{\@p@sheight}{\@bbw}{\@bbh}
		%\typeout{ \@p@sheight * \@bbw / \@bbh, = \@result }
		\edef\@p@swidth{\@result}
		%\typeout{w from h: width is \@p@swidth}
}
\def\compute@hfromw{
		% computing : height = width * (bbh / bbw)
		\in@hundreds{\@p@swidth}{\@bbh}{\@bbw}
		%\typeout{ \@p@swidth * \@bbh / \@bbw = \@result }
		\edef\@p@sheight{\@result}
		%\typeout{h from w : height is \@p@sheight}
}
\def\compute@handw{
		\if@height 
			\if@width
			\else
				\compute@wfromh
			\fi
		\else 
			\if@width
				\compute@hfromw
			\else
				\edef\@p@sheight{\@bbh}
				\edef\@p@swidth{\@bbw}
			\fi
		\fi
}
\def\compute@resv{
		\if@rheight \else \edef\@p@srheight{\@p@sheight} \fi
		\if@rwidth \else \edef\@p@srwidth{\@p@swidth} \fi
}
%		
% Compute any missing values
\def\compute@sizes{
	\compute@bb
	\compute@handw
	\compute@resv
}
%
% \psfig
% usage : \psfig{file=, height=, width=, bbllx=, bblly=, bburx=, bbury=,
%			rheight=, rwidth=, clip=}
%
% "clip=" is a switch and takes no value, but the `=' must be present.
\def\psfig#1{\vbox {
	% do a zero width hard space so that a single
	% \psfig in a centering enviornment will behave nicely
	%{\setbox0=\hbox{\ }\ \hskip-\wd0}
	%
	\ps@init@parms
	\parse@ps@parms{#1}
	\compute@sizes
	%
	\ifnum\@p@scost<\@psdraft{
		\if@verbose{
			\typeout{psfig: including \@p@sfile \space }
		}\fi
		%
		\special{ps::[begin] 	\@p@swidth \space \@p@sheight \space
				\@p@sbbllx \space \@p@sbblly \space
				\@p@sbburx \space \@p@sbbury \space
				startTexFig \space }
		\if@clip{
			\if@verbose{
				\typeout{(clip)}
			}\fi
			\special{ps:: doclip \space }
		}\fi
		\if@prologfile
		    \special{ps: plotfile \@prologfileval \space } \fi
		\special{ps: plotfile \@p@sfile \space }
		\if@postlogfile
		    \special{ps: plotfile \@postlogfileval \space } \fi
		\special{ps::[end] endTexFig \space }
		% Create the vbox to reserve the space for the figure
		\vbox to \@p@srheight true sp{
			\hbox to \@p@srwidth true sp{
				\hss
			}
		\vss
		}
	}\else{
		% draft figure, just reserve the space and print the
		% path name.
		\vbox to \@p@srheight true sp{
		\vss
			\hbox to \@p@srwidth true sp{
				\hss
				\if@verbose{
					\@p@sfile
				}\fi
				\hss
			}
		\vss
		}
	}\fi
}}
\def\psglobal{\typeout{psfig: PSGLOBAL is OBSOLETE; use psprint -m instead}}
\psfigRestoreAt








\newcommand{\segment}[3]{{\tt (SEGM}( #1,#2) \then #3{\tt )}}
\newcommand{\EndLabel}[2]{{\tt (ENDLABEL}\;#1\;\;#2{\tt )}}
\newcommand{\Lexit}{\ensuremath{L_{{\mathrm exit}}}}

\newcommand{\IfThenElse}[3]
	{\ensuremath
	{\begin{array}{lcl} 
		\mbox{if   }\; #1\; \mbox{ then}\\
		\hspace{4ex} #2\\ 
		\mbox{else }\; #3 \end{array}}}

\newcommand{\RCBB}{RCBB}
\newcommand{\Exp}{\ensuremath{Exp}}

\newcommand{\alloc}[1]{\ensuremath{\mathtt{ALLOC}(#1)}}
\newcommand{\dealloc}[1]{\ensuremath{\mathtt{DEALLOC}(#1)}}
\newcommand{\Obs}[3]{\ensuremath{\mathsf{Obs}(#1,#2,#3)}}
%\newcommand{\Obs}[3]{\mathbf{[}#1\; 
% \;\mbox{\raisebox{-0.4ex}{{\huge $\triangleright$}}}\;
% #2\; \| \;#3\mathbf{]}}
\newcommand{\IP}[3]{\ensuremath{\langle #1,#2,#3\rangle}}

\newcommand{\pbang}{\mathsf{p!}}

%\newcommand{\staged}[1]{\ensuremath{\mathsf{staged}\ldbrack #1 \rdbrack}}
\newcommand{\staged}[1]{\ensuremath{\mathcal{S}\ldbrack #1 \rdbrack}}
\newcommand{\unquote}{{\tt unquote}}

\newcommand{\bindM}{\;\star_{M}\;}
\newcommand{\unitM}{{\bf unit}_M}
\newcommand{\highlight}[1]{\begin{center} \fbox{#1} \end{center}}
\newcommand{\reason}[1]{\vspace{3ex}\noindent\fbox{{#1}}}
\newcommand{\dynwhile}{\ensuremath{\mathsf{dynwhile}}}
\newcommand{\whilePS}{\ensuremath{\mathsf{whilePS}}}
\newcommand{\FreshLabels}{\mathsf{FreshLabels}}
\newcommand{\FreshLabel}{\mathsf{FreshLabel}}
\newcommand{\FreshLoc}{\mathsf{FreshLoc}}

%\newcommand{\qed}{\noindent\(\sqcap\!\!\!\!\sqcup\)}
\newcommand{\NewIn}[2]{\mathsf{NewIn}(#1,#2)}
\newcommand{\dom}[1]{\ensuremath{\mathsf{dom}(#1)}}


%\newcommand{\assign}{\mbox{{\bf{\tt \large{:=}}}}}
%\newcommand{\assign}{\large{\mathbf{:=}}}
\newcommand{\assign}{\mathbf{:=}}

\newcommand{\restate}[5]{\label{#2-appendix}\noindent{\bf #1~\ref{#2} (#3)} {\em #4} #5}
\newcommand{\restateBlank}[5]{\label{#2-appendix}\noindent{\bf #1~\ref{#2} #3} {\em #4} #5}
\newcommand{\CF}[1]{\ensuremath{\mathsf{CF}\ldbrack #1 \rdbrack}}

\newcommand{\constexp}[1]{\ensuremath{\mathsf{constexp}(#1)}}
\newcommand{\AND}{\;\mathsf{AND}\;}

%\newcommand{\ObsArray}[3]
%{\ensuremath
%{\left[\begin{array}{lcl} 
%		 #1\;\mbox{\raisebox{-0.4ex}{{\huge $\triangleright$}}} \\
%\hspace{6ex} #2 \mbox{\raisebox{-0.25ex}{{\Large $\;\|$}}} \\ 
%		\hspace{12ex} #3 \end{array}\right] }}

\newcommand{\ObsArray}[3]
{\ensuremath
{(\mathsf{Obs}\left(\begin{array}{lcl} 
		 #1,\\
\hspace{6ex} #2,  \\ 
		\hspace{12ex} #3 \end{array}\right)) }}

\newcommand{\ObsArrayNP}[3]
{\ensuremath
{\mathsf{Obs}\left(\begin{array}{lcl} 
		 #1,\\
\hspace{6ex} #2,  \\ 
		\hspace{12ex} #3 \end{array}\right) }}

\newcommand{\ObsArrayNew}[3]
{\ensuremath
{\mathsf{Obs}\left(\begin{array}{lcl} 
		#1,\\
                \hspace{2ex}#2,  \\ 
		\hspace{2ex}#3 \end{array}\right) }}

\newcommand{\Pre}{\ensuremath{\mathsf{Pre}}}
\newcommand{\ifcmd}[2]{\ensuremath{\mbox{{\rm {\bf if }}} #1 \mbox{ {\rm {\bf then}} } #2}}
\newcommand{\whilecmd}[2]{\ensuremath{\mbox{{\rm {\bf while }}} #1 \mbox{ {\rm {\bf do}} } #2}}
\newcommand{\ifthenelsecmd}[3]{\ensuremath{\mbox{{\rm {\bf if }}} #1 \mbox{
{\rm {\bf then}} } #2 \mbox{ {\rm {\bf else}} } #3}} 

\newcommand{\LEQ}{\ensuremath{\;\mathsf{leq}\;}}
\newcommand{\srcleq}{\ensuremath{\mathsf{leq}}}
\newcommand{\boost}{\ensuremath{\mathsf{boost}}}
\newcommand{\W}[1]{\ensuremath{\mathcal{W}#1}}
\newcommand{\I}[1]{\ensuremath{\mathcal{I}#1}}
\newcommand{\ul}[1]{\underline{#1}}
\newcommand{\ajump}{{\tt ajump}}
%\newcommand{\equaldef}{=^{\hspace{-1.6ex}{\tiny\mbox{\raisebox{0.2ex}{$\triangle$}}}}}

\newcommand{\equaldef}{\stackrel{\triangle}{=}}

\newcommand{\nop}{\ensuremath{\mathtt{NOP}}}

\hyphenation{me-ta-com-put-ation me-ta-com-put-a-tion-based me-ta-com-put-ation---the}

\hyphenation{rep-res-ent-a-tion-al-ly-ind-ep-end-ent}

\newcommand{\Src}{\ensuremath{\mathsf{Src}\/}}

%\newarrow{Corresponds}{<}{-}{-}{-}{>}

\newcommand{\comment}[1]{}
%\newcommand{\comment}[1]{{\tiny {\rm {\bf (#1)}}}}

\newcommand{\invisible}[1]{}
%\newcommand{\invisible}[1]{#1}

\newcommand{\Value}{\ensuremath{\mathit Value}}

\newcommand{\implies}{\supset}
\newcommand{\F}{\ensuremath{\mathcal{F}}}
\newcommand{\typeMT}[2]{#1_{\mathsf{#2}}}

% New definitions
\newcommand{\hletexp}[2]{\ensuremath{let\;#1\;in\;#2}}
\newcommand{\hlambda}[2]{\ensuremath{\backslash #1 \mbox{{\tt ->}}\, #2}}
\newcommand{\hsem}[1]{\ensuremath{\standard{\mathtt{#1}}}}

\def\lastname{Sheard, Harrison, and Hook}
\begin{document}
\begin{frontmatter}
  \title{Modeling Fine Control of Demand in Haskell} 
      \author{Tim Sheard\thanksref{ALL}\thanksref{myemail}}
      \author{William Harrison\thanksref{coemail}}
      \author{James Hook\thanksref{triemail}}
      \address{Pacific Software Research Center\\
               Oregon Graduate Institute\\
               Beaverton, Oregon}                
      \thanks[ALL]{The work described here was supported by NSF Grant CDA-9703218, 
                   the M.J. Murdock Charitable Trust and the Department of Defense.} 
      \thanks[myemail]{Email: {\texttt{\normalshape sheard@cse.ogi.edu}}} 
      \thanks[coemail]{Email: {\texttt{\normalshape wlh@cse.ogi.edu}}}
      \thanks[triemail]{Email: {\texttt{\normalshape hook@cse.ogi.edu}}}
    
    
\begin{abstract} 
The lazy functional language Haskell can be viewed at many levels.  At the 
highest level it can be manipulated as if it were the call-by-name lambda 
calculus.  Slightly below this level is the model of the call-by-need 
lambda calculus, where control over duplication of computation becomes 
explicit. Underneath this level is a more detailed model that provides 
precise control over the evaluation of data constructors and 
destructors.  Expert Haskell programmers move between these three levels 
with ease, often making detailed operational arguments about program 
representations when discussing the third level.  One of the attributes 
that makes Haskell so successful is the extraordinary semantic coherence 
between these three levels.  This paper introduces a compositional 
meta-circular interpreter for Haskell that articulates the ``third level" of 
semantics, in which the fine control of evaluation in data construction and 
analysis is made precise.

\end{abstract}
\end{frontmatter}

\section{Introduction}\label{intro}

Haskell is a large, lazy, industrial-strength language, that is usable and
efficient across a wide variety of domains. We commonly refer to Haskell as
a {\em lazy} language, but technically it is really {\em non-strict}
because it contains constructs that make small changes to standard lazy
evaluation. These features make Haskell both more efficient
and harder to understand.

In this paper we present a meta-circular\cite{Steele-Sussman78b,Abelson-Sussman-Sussman96,Kamin,Friedman}
interpreter (of Haskell, written in Haskell) which provides a unified model of
these constructs and their interactions with Haskell's other features. Our
interpreter is a compositional monadic interpreter written in the style of
Liang, Hudak, and Jones \cite{LiangHudak}.

In particular, we model nested patterns, several different forms of lazy
pattern matching, guards and {\tt where} clauses on equations, strict
constructor functions, the {\tt newtype} datatype definition facility, and
the {\tt seq} operator. All these features make subtle changes to the
default lazy evaluation strategy of Haskell, and all for compelling reasons.
But, these slight changes are not always easy for programmers to
understand. A simple concise model of these features, understandable by
ordinary Haskell users is of significant benefit to the Haskell community.

\subsection{Rationale}

As part of the {\em Programatica}\cite{programatica} project at the Oregon Graduate
Institute, we are attempting to develop both a theory of Haskell programs, and some
automated tools for manipulating this theory. An important part of our work is to develop a
logic with which to manipulate Haskell terms. In order to develop sound rules of inference
for the logic, we need to know the meaning of the terms. This led us to the literature,
which to our surprise, was lacking in formal descriptions of the {\em big picture} of
Haskell. There are plenty of papers about particular features of Haskell, (its
laziness\cite{PeytonJone92a}, its class system\cite{JonesMark93,Hall:1994:TCH}, etc.) but
very little work which unifies the {\em essence of Haskell} with all the fine-control
mechanisms that have been added and refined over the years. Just how does Haskell differ
from the lazy lambda calculus\cite{Stoy,Gunter}?

This work started as an exercise to satisfy our own curiosity. Could we write a
Haskell interpreter that successfully modeled {\em all} of the control structures of
Haskell, yet remain simple and concise? Soon, the exercise became more than just a
simple toy. It became a model by which we judged much of our other work. We did not
anticipate that the model could be as simple and elegant as it turned out to
be. We quickly gained a deep appreciation for the continuity and semantic coherence of
all of the control features of Haskell. In addition to being simple, concise, and
elegant the model is useful! We wish to share it with the Haskell community at large.

Two important questions one might ask are {\em Why use Haskell to describe
Haskell?} and {\em What features should be described?}  We use Haskell because
many of those most interested in a semantics for Haskell features already have a
good, although not precise, appreciation for the meaning of Haskell programs.
Thus, those interested in Haskell, can use our interpreter as a concise
description or standard to which other descriptions (or implementations) can be
compared. It also provides a common language in which extensions and additions
to Haskell can be described precisely and concisely.

What criteria should be used to select the features of Haskell to be represented in our
interpreter? Our goal was (and still is) to describe all of Haskell, yet the features of
Haskell that deal with modules, type classes, comprehensions, and the {\tt do} notation,
depend upon programs being annotated with type information. The work reported in this paper
deals with a core fragment that does not require type annotations for unambiguous
interpretation. This fragment includes all the control structures in Haskell. We see this
work as a key step towards a denotational semantics for all of Haskell.

What strategy we should use in writing our interpreter? The Haskell'98
report\cite{Haskell98} uses a strategy of translating many complex constructs into a
simpler core language\cite{newcore}, that is really just a slightly extended
lambda-calculus. We feel that the translation based approach, while useful, is semantically
problematic. Many of the translation schemas in the Haskell Report rely on two devices.
First, the introduction of new variables; and second, the restructuring of code to avoid
duplication (usually to maintain sharing and to avoid code blowup). The semantics of
languages with binding mechanisms are very complex. When one defines a language feature by
a translation that introduces new variables, one leaves the well understood world of
domains and continuous functions, and moves into a much more complicated world. Recent work
on languages with binding mechanisms suggests that these kinds of transformations are by no
means trivial to model correctly\cite{Moggi,Pitts00}. For this reason we eschew such
techniques when there are relatively simple alternatives. A significantly larger core, but
with many fewer and simpler translations is more apt for describing the control features of
Haskell. These features were introduced to give the programmer explicit control over
evaluation in a {\em high-level} manner. It is incongruous to give these features meaning
in terms of a translation into the very same convoluted programs the features were meant to
replace.

One of the reasons for our research is to provide a concise model of Haskell. We
wish to point at a particular line of code in our model, and proclaim, if you
change {\em that line}, then {\em this change} will become evident in the model.
We particularly wanted to be able to do this with the control features of
Haskell. Our interpreter is meant to provide a mechanism to make such fine
distinctions.

Our goals for this research are three-fold. First to collect together in one place a formal
description of a larger subset of Haskell than has been previously described. Second, to
provide a concise description that can be used to communicate about Haskell
implementations. Last, we view this work as the first step in a program to define a formal
logic of Haskell programs, ultimately to be embedded into a large automated
toolset\cite{programatica}.


\section{A Model for Haskell}

This section presents a call-by-name model of Haskell as a
compositional meta-circular monadic interpreter; a call-by-need
strategy is discussed in Section~\ref{callbyneed}.  Call-by-name
is appropriate for a theory of program correctness, but may be
inappropriate for a theory of program optimization. A call-by-need
approach tries to model sharing, and thus resources consumed.  Based
on the work of Sabry and Liang we expect that the primary difference
between the call-by-name and call-by-need interpreters will be in the
underlying monad of computation selected\cite{JFP::Sabry1998}.

When Moggi\cite{MoggiE90a} introduced monads as a technique to structure semantics, he
introduced the distinction between a value of a type and a computation of a given type.
Using this basic distinction, Danvy\cite{Danvy} identified two basic forms of monadic
interpreters for functional languages:  the Kleisli interpreter that models functions as
mappings from values to computations, and the Reynolds\cite{Reynolds} interpreter that
models functions as transformations on computations.  The difference between the two is
that the Reynolds interpreter makes the evaluation strategy (call-by-value vs.
call-by-name) of the language being modeled explicit without depending upon the evaluation
strategy of the meta-language. A Reynolds interpreter can be used to express either
call-by-name or call-by-value (or something in between like Haskell), irregardless of the
evaluation strategy of the meta-language. Only a lazy meta-language can implement a
call-by-name strategy using a Kleisli interpreter.

The meta-circular interpreter developed here reflects the structure of
the Reynolds interpreter in several ways.  First, the notion of value
of the interpreter corresponds to terms in weak-head-normal form, with
computations ({\tt M Value}), rather than base values (simply {\tt
Value}), embedded recursively.

{\small
\begin{verbatim}
data Value                   -- Models values in Haskell
   = Z Integer                -- Scalars
   | FV (M Value -> M Value)  -- CBN functions
   | Tagged Name [M Value]    -- Algebraic structured data
   | TupleVal [M Value]       -- Tuples
\end{verbatim}
}

Secondly, the environment maps names to computations, not to values.
That is, it is possible to put a computation in the environment that
may not even have a weak-head-normal form value.  This is expressed:

{\small
\begin{verbatim}
type Env = Name -> M Value        -- Maps Names to computations
type EnvFrag = [(Name,M Value)]   -- Intensional environment fragment
data Error a = Ok a | Err String  -- A run-time error

newtype M a = M (Env -> (Error a)) -- M is an environment monad
                                    -- with errors. See Figure 1.
\end{verbatim}
}

Finally, the types of the meaning functions defining the interpreter yield
computations (as opposed to values) as their ranges. These functions have the
following structure. For each syntactic category of the language, {\tt T},
there is a function, {\tt mT :: T -> t1}. We say that {\tt mT} computes the
{\em meaning of} {\tt T}, and associates the type {\tt t1} with this meaning.
The type {\tt t1} often has the form ({\tt M t2}) or ({\tt M t3 -> M t4}),
where {\tt M} is the computation monad.

To be concrete, consider meaning functions for the datatypes defined in
Figure \ref{AST}. We have divided our Haskell subset into many syntactic
categories, three of which are called {\tt Exp}, {\tt Dec}, {\tt Pat} (for
expressions, declarations, and patterns).  Each has a corresponding meaning
function:

{\small
\begin{verbatim}
mExp :: Exp -> M Value
mDecs :: [Dec] -> M a -> M a
mPat :: Pat -> M Value -> M (Maybe EnvFrag)
\end{verbatim}
}

We have found the following analogy to be a useful intuition for thinking about such
meaning functions. Functions such as {\tt mExp :: Exp -> M Value}, can be thought of as
computation producers. The function {\tt mExp} produces a computation from an {\tt Exp}.
This computation computes a result of type {\tt Value}.

Functions such as {\tt mDecs :: [Dec] -> M a -> M a} can be thought of as producing
computation transformers. The term {\tt (mDecs ds):: M a -> M a} is a computation
transformer. Taking one computation as input, and producing a different but related
computation that adds the bindings indicated by {\tt ds} to the original computation.
It adds the necessary computational steps to extend the environment with the meaning of
the values bound in {\em ds}. The function {\tt mDecs} is used to give meaning to the
Haskell expression {\tt (let {\em ds} in {\em e})} by transforming the translation of
{\em e} as follows: {\tt (mDecs {\em ds} (mExp {\em e}))}.


A compositional interpreter must have the following additional property: the
meaning of any term in a syntactic category {\tt T} must depend, in a purely
functional manner, only on the meaning of its subterms. Monadic interpreters
have the ability to cleanly separate different concerns\cite{Wadler92}, and
are remarkably concise and reusable.  Haskell has introduced explicit support
for monadic abstractions in the  {\tt do} notation. The {\tt do} notation
combines monadic expressions (computations) into larger monadic expressions
(bigger computations). The {\tt do} notation both composes computations, and
makes clear the order in which the computations are to be run.

\begin{figure}
{\small
\begin{center}
{\bf Definition of monad {\tt M}.}
\end{center}
\begin{verbatim}
instance Monad Error where
   return = Ok
   (>>=) x f = case x of
                (Ok v) -> f v
                (Err msg) -> (Err msg)

instance Monad M where
   return a = M (\rho -> return a)
   (>>=) x f = M g
     where g rho = do { v <- (deM x) rho ; deM (f v) rho }
           deM (M x) = x

-- Non-standard Morphisms
lift :: Error a -> M a
lift ec = M (\ _ ->  ec)

raise :: String -> M a
raise = \msg -> lift (Err msg)

rdEnv :: M Env
rdEnv = M (\rho -> return rho)

inEnv :: Env -> M a -> M a
inEnv rho (M x) = M (\ _ -> x rho)

instance Functor M where
    fmap f x = do { x' <- x; return(f x') }
\end{verbatim}
\caption{A computation has type{\tt (M Value)}.
{\tt M} is an environment monad with errors.}\label{Monad}
}
\hrule
\end{figure}



\subsection{Which Monad?}

The definition of our monad in Haskell can be found in Figure \ref{Monad}. We call
our computational monad {\tt M}. It is an environment monad with errors. We define
it in the monad transformer style\cite{LiangHudak,LiangThesis,Espinosa95PHD}.
First defining a pure {\tt Error} monad, and then laying the environment features
on top.

The monad abstracts away the additional structure necessary to build compositional
interpreters. In Haskell, this additional structure must deal correctly with two
features: suspended computation, and the possibility of failure.

In Haskell, there are several distinct kinds of failures. Only one of which is
modeled in the monad {\tt M}. 

\begin{enumerate}

\item The first kind of failure, which is modeled within the monad, arises from
      run-time errors, such as division by zero, and non-exhaustive pattern match
      coverage. Taking the {\tt head} of the empty list is an example of this kind
      of failure.

\item The second kind of failure, stems from non-terminating computations. This
      kind of failure is captured by the model itself not terminating when
      given a non-terminating Haskell program as input.

\item The third kind of failure, stems from pattern-matching failure in a context
      where the failure can be caught, and control then proceeds to the next
      {\tt Match} of a case, or the next {\tt Clause} of a multi-line function
      definition. We model this outside the monad, as a computation of type
      {\tt M(Maybe a)}. We call such computations {\em partial} computations.
      A partial computation can become a failure of the first kind, if it occurs
      in a context with no {\em next} {\tt Match} or {\tt Clause}.
\end{enumerate}

In our semantics, a failure of the first kind is reported by using the non-standard
morphism {\tt raise}. The monad propagates failures once they are raised.

Suspensions are built using an environment. The environment component of {\tt M} is a
function ({\tt Name -> M Value}), which binds the free variables found in terms to
computations. This environment is used to build suspended computations. Intuitively, a
suspended computation is like a thunk or closure. It contains two components: a set of
instructions to execute, and a set of bindings for the free variables that appear in the
instructions. Suspensions are built at the monadic level using a classic pattern. The
nonstandard morphism {\tt rdEnv} is used to capture the current static mapping, and the
morphism {\tt inEnv} takes a captured mapping and a computation, and is used to create a
new computation that will run under the captured mapping.

The structure of {\tt Values} and the monad {\tt M} are conventional. We assume the
reader has some familiarity with monads\cite{Wadler92,Wad92:monads,Wadler:1995:MFP}
and interpreters\cite{LiangHudak,Espinosa95PHD,DSL99*81}


\subsection{Monadic Control Structures}

A nice property of monadic computation in Haskell is that it is easy to capture
patterns of control and abstract them into functions at the monadic level. A
familiar pattern that occurs many times in our semantics is a monadic conditional.
While interpreting a syntactic category, some computation, called the {\em
discriminant}, is evaluated lazily, just enough to make a choice between one of two
alternate paths. This pattern occurs about a dozen times in our semantics. These
fall into one of three categories, depending upon the {\em type} of the
discriminant.

When the discriminant has type ({\tt M Value}), and that value is representing
a {\tt Bool} type, as it would in an if-then-else or guarded-body expression,
we have a simple lifting of the {\tt if-then-else} expression of Haskell. We
capture this pattern with the function {\tt ifV}.
{\small
\begin{verbatim}
ifV :: M Value -> M b -> M b -> M b
ifV test thenM elseM =
  do { v <- test
     ; case v of
         Tagged "True" [] -> thenM 
         Tagged "False" [] -> elseM 
         other -> raise "Not a Bool in ifV" }
\end{verbatim}
}


A second kind of choice is captured by the function {\tt caseM}. It is used when
describing pattern matching. Pattern matching is inherently partial. For example in
a {\tt case} expression, when a pattern does not match, control passes to the next
arm of the case. When matching succeeds, additional information is captured: the
bindings for the variables bound by the pattern. This type of behavior is captured
by a discriminant with type {\tt M(Maybe a)}. The two arms of the conditional
correspond to the computation producing {\tt Nothing} or {\tt (Just x)}. The arm
for the {\tt (Just x)} case should be a function that produces a computation when
applied to {\tt x}. {\small

\begin{verbatim} 
caseM :: M (Maybe b) -> M c -> (b -> M c) -> M c 
caseM test noMatch matchF = 
  do { vbl <- test ; 
       case vbl of 
         Nothing -> noMatch
         Just x -> matchF x }
\end{verbatim} }

The last form of conditional is called {\tt hasTag} and is used to do dynamic
matching of tags in algebraic structured data. This kind of conditional compares a
statically known tag against the tag of a dynamically produced {\tt Tagged} value.
If the tag matches, take one path, if it does not match, take the other path. In
either case, additional information becomes available that is useful on the chosen
path. If the match is successful, the list of pairs binding variables
(representing the variables bound by the pattern) to suspended computations
(representing the data stored in the algebraic type), becomes known. If the tags
are not the same, then the non-matching tag name (useful for reporting failure)
becomes known. Thus the arms of {\tt hasTag} are both represented by functions.

{\small
\begin{verbatim}     
hasTag :: String -> M Value -> 
          ([M Value] -> M b) -> (String -> M b) -> M b     
hasTag tag arg mPatf nomPatf =
  do { v <- arg
     ; case v of
         (Tagged t mvals) -> if t==tag then mPatf mvals else nomPatf t
         other -> raise "Not a Tagged Value in hasTag" }
\end{verbatim}
}


 
\begin{figure}[t]
\begin{center} {\small {\bf Haskell Subset Abstract Syntax}}
\end{center}
{\small 
\begin{verbatim}
type Name = String
data Op = Plus | Mult | IntEq | IntLess
data LS = Lazy | Strict deriving Eq

data Pat 
  = Pconst Integer            -- { 5 }
  | Pvar Name                 -- { x }
  | Ptuple [Pat]              -- { (p1,p2) }
  | Pcondata Name [Pat]       -- data T1 = C1 t1 t2; {C1 p1 p1} = e 
  | Pnewdata Name Pat         -- newtype T2 = C2 t1;  {C2 p1} = e
  | Ptilde Pat                -- { ~p }
  
data Exp 
  = Var Name                  -- { x }
  | Const Integer             -- { 5 }
  | App Exp Exp               -- { f x }
  | Abs [Pat] Exp             -- { \ p1 p2 -> e }
  | TupleExp [Exp]            -- { (e1,e2) }
  | ConApp Name [(Exp,LS)]    -- data T1 = C1 t1 t2; p = {C1 e1 e2}
  | NewApp Name Exp           -- newtype T2 = C2 t1; p = {C2 e1}
  | Seq Exp Exp               -- { seq e1 e2 }               
  | Bin Op Exp Exp            -- { e1 + e2 }
  | Cond Exp Exp Exp          -- { if e1 then e2 else e3 }
  | Let [Dec] Exp             -- { let x=e1;   y=e2 in e3 }
  | Case Exp [Match]          -- { case e of m1; m2 }
  
type Match = (Pat,Body,[Dec]) -- case e of { pat -> body where decs } 
data Dec 
  = Fun Name [Clause]         -- { f p1 p2 = b where decs }
  | Val Pat Body [Dec]        -- { p = b where decs }

type Clause = ([Pat],Body,[Dec]) -- f { p1 p2 = body where decs }
data Body
  = Guarded [(Exp,Exp)]       -- f p { | e1 = e2 | e3 = e4 } where ds
  | Normal Exp                -- f p = { e } where ds
\end{verbatim}
}
\caption{{\footnotesize The comments following each constructor function are meant to
illustrate the concrete syntax represented by that constructor. Each comment
has the form {\em context\{ example \}} where {\em context} is possibly empty.
The {\em context} is supposed to place the {\em example} (which could be
a possibly ambiguous program text fragment) into perspective. In the examples, we use semicolon
(;) where the offsides rule would normally be used.}}\label{AST}
\hrule
\end{figure}

\section{The Abstract Syntax}

Haskell is a rich language with many features. Our subset identifies 9 different
syntactic categories. These categories include names, operators, strictness
annotations, matches, clauses,  bodies, expressions, declarations, and patterns.

In Figure \ref{AST} we display the {\tt data} definitions in Haskell that represent
our abstract syntax. Our definitions are conventional and we have used
comments to relate the abstract syntax to the concrete syntax of Haskell. The
missing syntax, necessary to complete a Haskell definition, has mostly to do with
the module system, classes, list comprehensions, and the do notation.


\section{The Meaning of Expressions}

The heart of a functional language like Haskell is the expression language. The meaning
of a Haskell program is given in terms of the meaning of the expressions it contains.
In our semantics, the meaning of a Haskell program is a computation producing a {\tt
Value}.

The rest of this section is a detailed explanation of the function {\tt mExp}. It is
divided into subsections. Each subsection describes a few related clauses making up
the definition of {\tt mExp} along with the auxiliary functions necessary for those
clauses. These auxiliary functions supply meaning for the other syntactic categories
of Haskell.

For pedagogical reasons, the text of the paper breaks the definition into a few clauses
at a time. An actual implementation would need to collect them all together in a
single place.

\subsection{Simple Expressions}


{\small
\begin{verbatim}
mExp :: Exp -> M Value
mExp (Var n) = do { rho <- rdEnv ; rho n }
mExp (Const i) = return (Z i)
mExp (TupleExp es) = 
   do { rho <- rdEnv 
      ; return (TupleVal (map (inEnv rho . mExp) es)) }
mExp (Cond l1 l2 l3) = ifV (mExp l1) (mExp l2) (mExp l3)
\end{verbatim}
}

The meaning of a variable is obtained by extracting the current environment ({\tt rho})
and then applying it to the variables name. It is useful to think of the {\tt do}
notation as building a computation. Lookup the computation stored in the
environment {\tt rho}, and return it.

Constants are simply turned into values and returned. No complicated translation
necessary.

In the case for tuples, the laziness of Haskell first becomes apparent. Each of the
subexpressions (the {\tt es}), must be suspended, and the resulting list of
computations is then wrapped by the {\tt TupleVal} constructor. To suspend a
computation one must first grab the current environment ({\tt rho}), translate the
expression, then apply the computation transformer {\tt (inEnv rho)} to enable the
computation to find the correct environment when it is later run. This is repeated
for each sub-expression in the tuple. We will see this pattern many times. One
cannot tell when, or where, a computation must be run, so it is suspended before
being embedded inside a {\tt Value}.
 
Conditional expressions, are easily translated using the control operator {\tt ifV}.

\subsection{Application and Abstraction}

Function application and abstraction are the key to any functional
language. Care must 
be taken to implement Haskell's laziness correctly.

{\small\label{CBNapplication}
\begin{verbatim}
mExp :: Exp -> M Value
mExp (App f x) =
   do { rho <- rdEnv
      ; (FV f1) <- mExp f
      ; f1 (inEnv rho (mExp x)) }
mExp (Abs ps l) = arrow ps (mExp l)
\end{verbatim}
}

To compute the meaning of an application, grab the current environment, evaluate
the function part ({\tt f}), to obtain a function value ({\tt FV f1}). Then apply
{\tt f1} to the suspended argument. The argument is suspended by first translating
{\tt x}, then transforming this computation to run in the current environment using
{\tt inEnv}. This is the {\em spot} where we specify a {\em Reynolds Style}
interpreter\cite{Reynolds}.

As we saw in the {\tt App} case the meaning of a function is a {\tt Value} of the form
{\tt FV f}, where {\tt f} is a function with type {\tt (M Value -> M Value)}. The meaning
of an abstraction must compute such a value. This is accomplished by translating the
body of the abstraction ({\tt l}), then applying the computation transformer {\tt
arrow}.

{\small
\begin{verbatim}
arrow :: [Pat] -> M Value -> M Value
arrow [] b = b
arrow (p:ps) b =
   do { rho <- rdEnv
      ; let f v = 
             caseM (mPat p v) 
                   (raise "mPat Error in Fun")
                   (\ vl -> inEnv (addBindings rho vl) (arrow ps b))
      ; return(FV f)
      }
\end{verbatim}
}

The key to understanding {\tt (arrow [p1, ... pn] body)} is that it builds a
nested sequence of {\tt FV} objects {\tt FV(\ v1 -> ... (FV \ vn -> body))}
where {\tt n} is the length of the list of patterns. The body of each of the
{\tt FV}s is a suspended computation, which incrementally adds the bindings to
the variables found in each {\tt pi}. The hard work is performed by the
transformer {\tt mPat}. The meaning of binding constructs, like patterns, in our
semantics is always associated with the production of an {\tt EnvFrag}. An {\tt
EnvFrag} is a list pairing {\tt Name}s to computations. An {\tt EnvFrag} can be
used to extend an {\tt Env} using the function {\tt addBindings :: Env ->
EnvFrag -> Env}.

\subsection{Patterns}\label{pattern}

Pattern matching nested patterns is a challenging problem when describing the
semantics of Haskell. We consider our treatment of nested patterns one of the
major contributions of this paper. Patterns can occur in several different
syntactic contexts. In lambda expressions ({\tt \ p1 p2 -> e}), in let
expressions ({\tt let p = e1 in e2}), in matches ({\tt case e1 of p -> e2}), and
in clauses in multi-line function definitions ({\tt f p1 p2 = e}). Patterns also
appear as sub-patterns inside other patterns.

Patterns are used in two distinctly separate semantic ways in Haskell. The way a
pattern is used is decided by the syntactic context in which it appears. We call
the two semantic meanings of patterns, {\em optimistic lazy} patterns, and {\em
discriminant strict} patterns. Optimistic lazy patterns occur in {\tt let}
expressions and where patterns are annotated as lazy (\verb+~p+). Discriminant
strict patterns occur in all other contexts. 

The two different meanings of patterns produce different kinds of transformers.
Both meanings of a pattern can be thought of as computation transformers. They
are used to transform the computation of the expression appearing in the
scope where the variables of the pattern are bound.

The function {\tt mPat} defines the meaning of discriminant strict patterns. The
function {\tt mPatLazy} defines the meaning of optimistic lazy patterns. Both
meanings for patterns are {\em lazy} in some sense. Discriminant strict patterns
are lazy in that they force only enough evaluation to determine if the
pattern matches. Optimistic lazy patterns are lazy in that they never force
any evaluation at all. They are optimistic in the sense that it is assumed that
the pattern will match, and computation proceeds on the assumption that it will.
If later events invalidate this assumption, then a failure of the first type
will be raised.

The types of the functions {\tt mPat :: Pat -> M Value -> M(Maybe EnvFrag)} and
{\tt mPatLazy :: Pat -> M Value -> EnvFrag} encode this information. 

The function {\tt mPat} gives meaning to discriminant strict patterns. The
partiality of discriminant strict patterns shows up as the {\tt Maybe} type.
Because discriminant strict patterns may force computation in the discriminant
the resulting {\tt M(Maybe EnvFrag)} is a monadic computation. The partiality of
{\tt mPat} is used as a signal to advance to the next {\em match} of a {\tt
case}, or the next {\em clause} of a multi-line function definition.

The function {\tt mPatLazy} gives meaning to optimistic lazy patterns. It
optimistically returns an {\tt EnvFrag} without forcing any evaluation. It only
forces evaluation of the discriminant when one of the variables bound in the
pattern is demanded. If the value can't be matched to the optimistic pattern a
failure of the first type is raised. Failure is delayed. Advancing to the next arm
in a multi-arm {\tt case} guarded by a lazy pattern is not possible, since an
arm guarded by an optimistic lazy pattern is irrevocably chosen. This is illustrated in the table
below. The first 3 lines of the table correspond to {\tt mPat} matching, and the
last two lines correspond to {\tt mPatLazy} matching.

\[
{\small
\begin{tabular}{|l|c|l|} \hline
{\em example} & {\em value} & {\em comment} \\ \hline
\verb+(\ x -> 3) bottom+  & 3 & succeeds  \\ \hline
\verb+(\ (x:xs) -> 3) bottom+  & bottom & diverges  \\ \hline
\verb+(\ (x:xs) -> 3) []+ & Failure & match failure \\ \hline
\verb+let x:xs = [] in 3+  & 3 &  succeeds \\ \hline
\verb+case [] of { ~(x:xs) -> x; [] -> 3 }+ &  Failure & delayed error \\ \hline
\end{tabular}
}
\]

The transformer, {\tt (mPat p)}, has type {\tt (M Value -> M (Maybe EnvFrag))}, and
transforms its translated argument into another computation which (at run-time of the
computation) computes a {\tt Maybe} type. The {\tt Maybe} type can be observed dynamically
and different courses of action can be taken. The result of a successful match is {\tt
(Just (x::EnvFrag))}. Recall that\\ \verb+type EnvFrag = [(Name,M Value)]+.\\ One can
think of an {\tt EnvFrag} as a dispatch table, pairing each variable with a computation
which needs to be executed to get that variable's value. These computations may be
non-trivial, as only a small amount of the computation necessary to produce the value that
the pattern was matched against may have been forced.

We now give the definition of {\tt mPat}. As for {\tt mExp} we will discuss
it a few clauses at a time.

{\small
\begin{verbatim}
mPat :: Pat -> M Value -> M (Maybe EnvFrag)
mPat (Pvar x) arg = return (Just [(x,arg)])
mPat (Pconst i) arg = 
    do { (Z v) <- arg
       ; if v==i 
            then return(Just [])
            else return Nothing
       }
\end{verbatim}
}

Matching a variable never forces any computation, and always succeeds, so a simple list
with a single pair is immediately returned. To match a pattern constant, the argument
must be forced so that its value can be compared against the constant in the pattern.

Tuples and structured data have much in common so we discuss both these cases
together. The key is to force only enough evaluation to match against
the pattern, and no more.


{\small
\begin{verbatim}  
mPat :: Pat -> M Value -> M (Maybe EnvFrag)
mPat (Ptuple ps) arg = 
    do { (TupleVal mvals) <- arg
       ; mfrags <- sequence(zipWith mPat ps mvals)
       ; return(allJust mfrags)
       }
mPat (Pcondata tag ps) arg = 
    hasTag tag arg
        (\ mvals -> do { mfrags <- sequence (zipWith mPat ps mvals)
                       ; return(allJust mfrags)
                       })
        (\ t -> return Nothing)

\end{verbatim}
}

For {\tt Ptuple}, {\tt arg} must be forced to get at the underlying tuple. Then each
sub-pattern is recursively matched against the sub-components of the tuple. Note that
if all the sub-patterns of the tuple are variables, no more computation will be
forced. If some are not variables, just those components of the argument tuple will be
forced. By zipping {\tt mPat} over the sub-patterns, and the suspended computations
that make up the body of the tuple, we compute a list of recursive results.
Each of the recursive results could either be {\tt (Just frag)} or {\tt Nothing}. If
they are {\em all} of the form {\tt (Just frag)} the subfragments are combined
into one large fragment.

{\small
\begin{verbatim}  
allJust :: [Maybe [a]] -> Maybe [a]  
allJust [] = Just[]
allJust (Just x : xs) = fmap (x ++) (allJust xs)
allJust (Nothing : xs) = Nothing  
\end{verbatim}
}

Structured data is similar. The {\tt hasTag} conditional operator forces
evaluation of the argument. If the tag matches, the first arm proceeds as for
tuples; If the tag does not match, the other arm immediately returns a partial
result.


{\small
\begin{verbatim}
mPat :: Pat -> M Value -> M (Maybe EnvFrag)
mPat (Pnewdata n p) arg = mPat p arg
mPat (Ptilde p) arg = return (Just (mPatLazy p arg))
\end{verbatim}
}

The constructor function introduced by a {\tt newtype} declaration acts like the
identity function. Thus to match a {\tt Pnewdata} pattern against an argument,
just match its sub-pattern against the argument. A similar rule is found in the
evaluation of {\tt newtype} constructor functions in the function {\tt mExp}.

A pattern annotated as lazy ({\verb+~p+), is treated as an optimistic lazy
pattern, regardless of its context. The function {\tt mPatLazy} always immediately
succeeds without forcing any computation at all. Since {\tt mPat} returns a {\tt
Maybe} result, and the call to {\tt mPatLazy} always returns just an {\tt
EnvFrag}, we must wrap the {\tt Just} constructor around the result of {\tt
mPatLazy}.

The meaning of an optimistic lazy pattern is a computation transformer. Since
matching an optimistic lazy pattern is never partial, the transformer transforms
the argument directly into an {\tt EnvFrag} (rather than a {\tt Maybe EnvFrag}),
binding variables to computations. Each such computation demands the argument when
one of the variables in the pattern is demanded. If the pattern does not match the
argument, these computations must incorporate a delayed match error. The key to
understanding {\tt mPatLazy} is to understand how the argument computation is
lazily incorporated into each computation in the {\tt EnvFrag}.


{\small
\begin{verbatim}
mPatLazy :: Pat -> M Value -> EnvFrag
mPatLazy (Pconst k) mv = []         
mPatLazy (Pvar x) mv = [(x,mv)]
mPatLazy (Ptilde p) mv = mPatLazy p mv
mPatLazy (Pcondata n ps) mv = concat(zipWith argN [1..] ps)
    where argN i p = mPatLazy p (subValue i)
          subValue i = do { vs <- hasTag n mv return error; vs !! (i-1)}
          error tag  = raise ("Tag " ++ n ++ " != " ++ tag)
mPatLazy (Pnewdata n p) mv = mPatLazy p mv                    
mPatLazy (Ptuple ps) mvals = concat(zipWith select [1..] ps)
    where select i p = mPatLazy p (proj i mvals)
          proj :: Int -> M Value -> M Value
          proj n phi = do { (TupleVal mvals) <- phi ; mvals !! (n-1)  } 
\end{verbatim}
}

For structured data, each sub-pattern may bind some variables, each of these patterns
pulls on the original argument when one of their variables is demanded. This happens
when calling {\tt subValue i} which in turn pulls on {\tt mv} through the call 
\verb+(hasTag n mv return error)+. A similar action takes place through the use of
the function {\tt proj} in the tuple case.

Optimistic lazy patterns with constant sub-patterns are worth thinking about
for a moment. An optimistic pattern always matches immediately, delayed
failure is raised only if pulling on the variables in the pattern discovers
that the value being matched does not have the structure of the pattern. An
optimistic lazy constant pattern always matches, and since it has no
variables, it never causes delayed failure. A constant pattern in an
optimistic lazy context acts as a wildcard pattern!

We now return to further discussion of the meaning of expressions.

\subsection{Constructor Application}

Constructor applications are evaluated in a manner much like tuples. The key
difference is the possibility that constructors can have some of their arguments
annotated as strict arguments. For example in Haskell, we might write\\ {\tt data
T = C String !Int}, to denote that the second argument to {\tt C} should be forced
into head normal form before the {\tt C} object is allocated in the heap. We
denote this in our abstract syntax by pairing each sub-argument with a strictness
annotation ({\tt LS}).

{\small
\begin{verbatim}
mExp :: Exp -> M Value
mExp (ConApp n es) = 
   do { rho <- rdEnv
      ; let f (e,Lazy) es   = 
              do { cs <- es; return ((inEnv rho (mExp e)):cs) }
            f (e,Strict) es = 
              do { c <- mExp e; cs <- es; return((return c):cs)}
      ; cs <- foldr f (return []) es
      ; return(Tagged n cs)
      }  
\end{verbatim}
}

Lazy arguments are suspended (using {\tt rdEnv} and {\tt inEnv}) in the current
environment, strict ones are evaluated immediately.  Inspecting the interpreter
shows that strict constructors store computations which are already in head-normal
form.

\subsection{Case Expressions}

In {\tt case} expressions we use the pattern matching transformers defined in
Section \ref{pattern}. A {\tt case} contains a {\em discriminant} ({\tt e}), and a
list of {\em matches} ({\tt ms}). A {\tt Match} consists of a pattern, a body, and
an optional list of local declarations\footnote{We discuss {\tt where} clause
declarations in Section \ref{decs}.}. The pattern of each {\tt Match} is matched
against {\tt e} in turn, returning the meaning of the body of the first successful
match. Since a match may be partial, our semantics must handle this properly. The meaning
of a match of the form ({\tt pat -> body}), is a transformer taking the meaning of
the discriminant to a computation returning a {\tt (Maybe Value)}. This has the
form {\tt (Just (mExp body))} if the pattern matches, and {\tt Nothing} if it does
not. For example, the case expression, {\tt (case x of m1; m2)} is given meaning
by the expression:

{\small
\[ \verb+(mMatch m1) (mExp x)+ \;[\!]\; \verb+(mMatch m3) (mExp x)+ \;[\!]\; \verb+raise "Match Failure"+ \] 
}

Here the right associative infix {\em fatbar} operator ($\;[\!]\;$), returns the first
(left most) operand that returns a {\tt Just} value. We make this precise by:

{\small
\begin{verbatim}
mExp :: Exp -> M Value
mExp (Case e ms) =  
   foldr (fatbar (mExp e)) (raise "Match Failure") (map mMatch ms)
  where fatbar arg mm next = caseM (mm arg) next return
\end{verbatim}
}

The meaning of a {\tt Match} is computed by {\tt mMatch ::  Match -> M Value ->
M(Maybe Value)}.  Like {\tt mPat}, {\tt mMatch} may force computation, and it may
also be partial.  Unlike a {\tt Pat},  {\tt mMatch} transforms a computation into
a {\tt (Maybe Value)} rather than a {\tt (Maybe EnvFrag)}.
The partiality of a {\tt Match} can stem from two causes.
Consider: 
\begin{verbatim} 
f (x:xs) | even x = 1 
\end{verbatim} 
The term {\tt (f [])} is partial because of pattern match failure; the term {\tt
(f [1])} is partial because of guard failure. 
{\small
\begin{verbatim}
mMatch :: Match -> M Value -> M(Maybe Value)
mMatch (p,body,ds) arg = 
   caseM (mPat p arg) 
     (return Nothing)
     (\frag ->  do { rho <- rdEnv
                   ; inEnv (addBindings rho frag) 
                             (mDecs ds (mBody body))
                   })                    
\end{verbatim}
}

The function {\tt mMatch} is basically an extension of {\tt mPat}, executing the
corresponding {\em body} in an environment extended with two additional sets of
bindings. The first set is the bindings found in the {\tt EnvFrag} computed by a
successful match, and the second is the bindings found in the {\tt where} clause {\tt
ds}. We discuss {\tt mDecs} in Section \ref{decs}.

A {\tt Body} is either a single expression or a list of guarded expressions.

{\small
\begin{verbatim}      
  where mBody :: Body -> M(Maybe Value)
        mBody (Normal e) = do { x <- mExp e; return(Just x)}
        mBody (Guarded guards) = mExpGuardList guards
        mExpGuardList :: [(Exp,Exp)] -> M (Maybe Value)
        mExpGuardList [] = return Nothing
        mExpGuardList ((g,b):gs) =
            ifV (mExp g)
                (do { x <- mExp b; return(Just x) })
                (mExpGuardList gs)
\end{verbatim}
}

The partiality of {\tt mBody} is made necessary by guarded expressions; it is
possible for all the guards to be partial. If this happens, the whole {\tt Match}
is partial as well, and control should pass to the next {\tt Match} in the
enclosing {\tt case} expression.


\subsection{Declarations and Let Expressions}\label{decs}

Sets of mutually recursive definitions appear in {\tt let} expressions, {\tt
where} clauses, and at the top-level of Haskell programs. The difference
between these is a subtle distinction of scope. The semantics of a set of
mutually recursive definitions is always a transformer. The distinction in
scope is made by the way the transformer is applied.

The meaning of a set of mutually recursive definitions is given by {\tt mDecs :: [Dec]
-> M Value -> M Value}. It is a transformer that transforms the computation in the scope
of the definitions into another that knows how to access those definitions. The difficulty
in defining a meaning function for a set of {\em mutually recursive} definitions, is the
{\em recursion}. The {\em use} of such a transformer is quite straightforward. For example,
the meaning of a {\tt let} expression can be given simply by:

{\small
\begin{verbatim}
mExp :: Exp -> M Value
Exp (Let ds body) = mDecs ds (mExp body)
\end{verbatim}
}

The {\em definition} of {\tt mDecs} is more problematic. The meaning of
binding constructs in our semantics is always associated with the production
of an {\tt EnvFrag}. An {\tt EnvFrag} is a list of pairs. Each pair consists
of a {\tt Name}, and computation. For a set of {\em mutually recursive}
declarations, the computations inside the {\tt EnvFrag} must include the
bindings for the names declared in the set. But this {\tt Environment} can
only be constructed once the {\tt EnvFrag} is complete. Thus we have a
chicken and egg problem. What comes first, the {\tt Env} or the {\tt
EnvFrag}?

Fortunately this problem is easy to solve using the recursion of the meta-language
Haskell. We expect the following recursive equation to hold of our semantics:

{\small
\begin{verbatim}
newEnv = addBindings oldEnv (makeDsFrag newEnv ds)
\end{verbatim}
}

Here {\tt makeDsFrag} produces an {\tt EnvFrag} given the {\tt Env} which
already contains bindings for the names declared in {\tt ds}. However this
equation is non-compositional. We give a compositional definition of {\tt newEnv}
below that satisfies this equation.

The function {\tt mDecs} builds a transformer that extends the environment with the
variables bound by the set of declarations. We make the recursion that solves the
chicken and egg problem explicit, by using an explicit {\tt fix} operator. The function
{\tt mDecs} works by grabbing the current environment, adding the bindings produced by
{\tt makeDsFrag} to make a new {\tt Env}, and using the {\tt inEnv} transformer to build
a suspension that will evaluate the original computation in this environment.

{\small
\begin{verbatim}
mDecs :: [Dec] -> M a -> M a
mDecs ds mExp =
     do { rho <- rdEnv
        ; let newEnv = fix (\ r -> addBindings rho (makeDsFrag r ds))
          in inEnv newEnv mExp
        }
\end{verbatim}
}               

The function {\tt makeDsFrag} is straight forward. There are two kinds of declarations,
{\tt Val} and {\tt Fun}. For each kind of declaration build a pair, pairing the names
bound with their associated computations. Recall that {\tt r} is an environment that
will contain the bindings we are now constructing.


{\small
\begin{verbatim}
makeDsFrag :: Env -> [Dec] -> EnvFrag
makeDsFrag r ds = foldr mDec [] ds
 where mDec (Val p body ds) frag = 
            mPatLazy p (inEnv r (mDecs ds (mBody body))) ++ frag
       mDec (Fun nm cls) frag = (nm,mExpCls cls) : frag
       mExpCls :: [([Pat],Body,[Dec])] -> M Value
       mExpCls [] = raise "no guard matches"
       mExpCls ((ps,body,ds):cls) = 
            arrow ps (inEnv r (mDecs ds (mBody body)))  
\end{verbatim}
}

Recall that the meaning of a declaration must never perform any computation,
only build new environments. Thus the use of {\tt mPatLazy} in the {\tt Val}
clause of the function {\tt mDec}. Note also the recursive use of {\tt mDecs}
to handle {\tt where} clauses inside each declaration.


\subsection{Newtype Constructor Application}

A {\tt newtype} constructor acts like the identity function, thus it is easy to define
the clause of {\tt mExp} for {\tt newtype} constructors:

{\small
\begin{verbatim}
mExp :: Exp -> M Value
mExp (NewApp n x) = mExp x
\end{verbatim}
}

\subsection{Miscellaneous expressions}

Finally, we come to the last few miscellaneous expression forms, the {\tt seq} operator,
and primitive binary operators. We assume that the primitive operators are strict in their
operands.

{\small
\begin{verbatim}
mExp :: Exp -> M Value
mExp (Seq x y) = do { mExp x; mExp y }
mExp (Bin op l1 l2) =
   do { (Z i1) <- mExp l1
      ; (Z i2) <- mExp l2
      ; return(mOp op i1 i2)
      }
 where mOp :: Op -> Integer -> Integer -> Value
       mOp Plus x y = Z(x + y)
       mOp Mult x y = Z(x * y)
       mOp IntEq x y = fromBool(x==y)
       mOp IntLess x y = fromBool(x < y)
       fromBool True = Tagged "True" []
       fromBool False = Tagged "False" []
\end{verbatim}
}

The {\tt Seq} operation needs some explanation. {\tt Seq} evaluates
its first operand, then evaluates and returns the value of its second
operand. The benefit of {\tt Seq} is only evident in a monad that models
the sharing of a call by need semantics. In such a model, one uses {\tt Seq}
to force computation of the first operand, so that subsequent shared demands
will perform only a small constant amount of computation. The most common
use of {\tt Seq} is to define a strict application operator.

{\small
\begin{verbatim}
strictApply f x = seq x (f x)
\end{verbatim}
}

For this to usefully, we would have to change our monad to encode sharing.

%%%%%%%%%%%%%%%%%%%%%%%%%%%%%%%%%%%%%%%%%%%%%%%%%%%%%%%%%%%%%%%%%%%%%%%

\section{Using the interpreter: adding call-by-need evaluation}\label{callbyneed}

With a few simple changes, the interpreter developed here can be made call-by-need
rather than call-by-name. This is a happy consequence of parameterizing our
interpreter by a monad structured with monad
transformers\cite{LiangHudak,LiangThesis}. The only changes necessary are adding a
heap structure to the monad {\tt M} and replacing the closures that are formed
through function application with self-updating thunks. In this section, we outline
these changes. Our treatment follows Liang's dissertation\cite{LiangThesis}
closely.

Previously, we defined {\tt M} as:
\begin{verbatim}
       M a = Env -> (Error a)
\end{verbatim}
The first change is to include a heap in the definition of {\tt M}:
\begin{verbatim}
       M a = Env -> (Error (Heap a))
\end{verbatim}
\noindent where:
\begin{verbatim}
       Heap a = Sto -> (Sto,a)
       Sto = (Loc,[(Loc,M Value)])
       Loc = Int
\end{verbatim}

\noindent {\tt Heap} is a state monad with heap structure {\tt Sto}. The first
component of {\tt Sto}, {\tt Loc}, represents the next free address in the heap. The
second component of {\tt Sto}, {\tt [(Loc,M Value)]}, represents the heap itself.
That is, each pair \verb+(l,thk)+ in the list, represents a thunk {\tt thk} stored at
location {\tt l}. With the new definition of {\tt M}, it is a simple matter to
specify operations {\tt (alloc : M Loc)}, {\tt (read : Loc -> M Value)}, and {\tt
(write : Loc -> M Value)} which allocate a new location in the heap, and read (and
write) from (and to) heap locations, respectively.

Now, the only necessary change to our interpreter is to replace closures with
thunks in the definition of function application:
{\small
\begin{verbatim}
mExp :: Exp -> M Value
mExp (App f x) =
   do { rho <- rdEnv
      ; (FV f1) <- mExp f
      ; loc <- alloc
      ; write loc (mkthunk rho loc (mExp x))
      ; f1 (read loc) }
where
     mkthunk r l phi = 
           do { v <- inEnv r phi ; write l (return v) ; return v }
\end{verbatim}
}

The first two lines of this code are identical to those in the
original call-by-name definition in Section \ref{CBNapplication}.
Then, a new location {\tt loc} is allocated in the heap, and a thunk for
the argument {\tt x} (i.e., {\tt mkthunk rho loc (mExp x)}), is written to {\tt
loc}. Now, rather than passing the closure {\tt (inEnv rho (mExp x))} as one
would in call-by-name, a reference to {\tt loc} is 
passed: {\tt (f1 (read loc))}. 

Defining application in this manner implements call-by-need evaluation. The first
time {\tt x} is evaluated within {\tt f}, the thunk {\tt (mkthunk rho loc (mExp x))}
will evaluate the closure {\tt (inEnv rho (mExp x))} to value {\tt v}, and will then
update itself by writing the unit computation {\tt (return v)} to {\tt loc}.
Successive evaluations of {\tt x} will share {\tt (return v)} rather than the
original closure, as call-by-name would.

To fully exploit the sharing of call-by-need in our interpreter we would have to rephrase
the pattern binding within matches in a case expression {\tt (case e of m1 ... mn)} so that
incremental evaluation of {\tt e} is performed.

%%%%%%%%%%%%%%%%%%%%%%%%%%%%%%%%%%%%%%%%%%%%%%%%%%%%%%%%%%%%%%%%%%%%%%

\section{Using the interpreter: evaluation subtleties}

\newcommand{\undef}{\ensuremath{\perp}}

Consider the following example taken from the Haskell'98
Report\cite{Haskell98} (page 43) which demonstrates the subtle
differences in evaluation between constructors introduced with {\tt
data},
{\tt data} with strictness annotations, and {\tt newtype}:
\begin{verbatim}
     data D1 = D1 Int                      d1 (D1 i) = 42
     data D2 = D2 !Int                     d2 (D2 i) = 42
     newtype N = N Int                     n (N i) = 42
\end{verbatim}

The differences in evaluation between {\tt D1}, {\tt D2}, and {\tt N} are then
illustrated by the following equivalences:

{\small
\begin{center}
\begin{tabular}{|c|c|}\hline
Haskell Terms & Equivalent to: \\
\hline
{\tt (d1 \undef)},\quad {\tt (d2 \undef)}, \quad {\tt (d2 (D2 \undef))} & \undef\\
\hline
{\tt (n \undef)},\quad {\tt (n (N \undef))},\quad  {\tt (d1 (D1 \undef))} & {\tt
42}\\
\hline
{\tt (N \undef)},\quad {\tt (D2 \undef)}, \quad ({\em but not} {\tt (D1 \undef)}) &
\undef\\
\hline
\end{tabular}
\end{center}
}

It is possible to point to lines in the code presented here which explain this
behavior! For example, exactly why is {\tt (D1 \undef)} not \undef, while {\tt (D2
\undef)} and {\tt (N \undef)} are? Just look at the clauses defining the various
flavors of constructor application:

%\vspace*{.5in}
\begin{minipage}[h]{6in}
{\footnotesize
\begin{verbatim}
         mExp (ConApp n es) =
           do { rho <- rdEnv
              ; let f (e,Lazy) es   =
{-1-}                 do { cs <- es; return ((inEnv rho (mExp e)):cs) }
                    f (e,Strict) es =
{-2-}                 do { c <- mExp e; cs <- es; return((return c):cs)}
              ; cs <- foldr f (return []) es
              ; return(Tagged n cs)
              }
{-3-}    mExp (NewApp n x) = mExp x

\end{verbatim}
}
\end{minipage}

%\vspace{.2in}

\noindent  {\tt (D1 \undef)} is evaluated as a lazy constructor application, and its
meaning (from the line marked ``{\small \verb+{-1-}+}'') is the non-\undef\/ suspension {\tt
(return (inEnv r \undef))}. Having a strictness annotation, {\tt D2} evaluates its
argument on the line marked ``{\small \verb+{-2-}+}'' with {\tt do \{ c <- \undef ; ...\}},
and so {\tt (D2 \undef)} produces \undef. A constructor introduced by {\tt newtype}
is not treated as a {\em bona fide} semantic function, as {\tt data} constructors
are, and as shown on line ``{\small \verb+{-3-}+}'', {\tt (N i)} and {\tt i} are semantically the same thing. Thus, {\tt (N
\undef)} and \undef\/ are equivalent.

Why does {\tt d1} behave strictly (i.e., {\tt (d1 \undef)} is equivalent to \undef),
while {\tt n} is lazy (i.e., {\tt (n \undef)} is equivalent to 42)? Just consider
the three lines defining pattern-matching on {\tt data} and {\tt newtype}
constructors:

{\footnotesize
\begin{verbatim}
     mPat (Pcondata tag ps) arg = hasTag tag arg  (...) (...)
     mPat (Pnewdata n p) arg = mPat p arg
     mPat (Pvar x) arg = return (Just [(x,arg)])
\end{verbatim}
}
\noindent It is enough to know that the discriminant {\tt hasTag} evaluates
its argument {\tt arg} to see that {\tt (d1 \undef)} is equivalent to \undef. 



%%%%%%%%%%%%%%%%%%%%%%%%%%%%%%%%%%%%%%%%%%%%%%%%%%%%%%%%%%%%

\section{Pattern-matching compilation is not just desugaring}

If a pattern {\tt p} is \(\mathtt{(C}\; \mathtt{t_1}\ldots \mathtt{t_n)}\) where
$\mathtt{t_i}$ are variables, and $\mathtt{C}$ is a constructor function, then
{\tt p} is a {\em simple} pattern. However, if one or more of the $\mathtt{t_i}$
are not variable patterns, then {\tt p} is a {\em nested} pattern.
Pattern-matching compilation\cite{PeytonJones86,SCP::JonesS1998} is typically performed as part of
the front-end (as it is in GHC and Hugs), because it yields more efficient
programs (see Chapter 5 by Wadler in \cite{PeytonJones86} for further details).
Figure~\ref{SyntacticSaccharin} shows an example of pattern-match compilation in
which the definition of a Haskell function {\tt nodups} with nested patterns is
transformed into a similar definition without nested patterns. One feature of this
transformation was the necessity of generating new variables {\tt x}, {\tt x'},
{\tt xs}, and {\tt xs'} along the way.


\begin{figure}
\hrule
\vspace{1.5ex}

{\small
\begin{verbatim}
nodups1 l =      -- Original with nested patterns:
  case l of
    [] -> []      
    [x] -> [x]
    (y:(x:xs)) -> if x==y then (nodups1 (x:xs)) 
                          else (y:(nodups1 (x:xs)))
                          
nodups2 xs'' =   -- After pattern-match compilation: 
  case xs'' of 
    [] -> []      
     x':xs' -> case xs' of
                [] -> [x']
                x:xs -> if x'==x then (nodups2 (x:xs)) 
                                 else (x':(nodups2 (x:xs)))
\end{verbatim}
}
%\vspace{-0.5cm}
\caption{Syntactic Saccharin}
%I checked the Merriam-Webster dictionary-this is the correct spelling of 
% ``Saccharin''
\vspace{1.5ex}
\hrule
\label{SyntacticSaccharin}
\end{figure}


Previous attempts\cite{PeytonJone92a,PeytonJones:1998:TBO} to define a denotational semantics for the core of
Haskell concentrate on the fragment of the language without nested patterns (the
kind of programs produced by pattern-match compilation). This semantics for
``unnested'' Haskell core could be extended simply to the full language by
defining the meaning of a term with nested patterns to be the meaning of its
compilation. For example, the meaning of {\tt nodups1} is identified with that of
{\tt nodups2}. Observe that, within this extended semantics, there is an implicit
reliance on the ability to generate fresh names just as {\tt x}, {\tt x'}, {\tt
xs}, and {\tt xs'} were generated. The implicit assumption in this approach
that pattern-match compilation is essentially a semantically irrelevant
elimination of syntactic sugar.

One drawback of this extended semantics is that it is no longer compositional. A
much more serious flaw, however, derives from the reliance on fresh name
generation within the pattern-matching compilation. Recent
developments\cite{Moggi,Gabbay99lics} in the semantics of staged languages reveal that the
structural consequences of including name generation within a denotational
semantics are considerable. This would have serious consequences for
developing a simple logic for Haskell programs.


%%%%%%%%%%%%%%%%%%%%%%%%%%%%%%%%%%%%%%%%%%%%%%%%%%%%%%%%%%%%%%%%%%%%%%%

%%%%%%%%%%%%%%%%%%%%%%%%%%%%%%%%%%%%%%%%%%%%%%%%%%%%%%%%%%%%%%%%%%%%%%%


\section{Conclusions}

Haskell is commonly referred to as a lazy functional language, but it is more properly
understood as a non-strict language because it contains features (patterns, the {\tt
seq} operator, etc.) which introduce strict perturbations of the default lazy
evaluation mechanism. These perturbations are important for practical reasons: expert Haskell
programmers may use strictness sparingly in their programs to avoid some of the
computational overhead associated with laziness without giving it up entirely.

However, this mixed evaluation order complicates Haskell from a
semantic point of view. We have modeled Haskell's control of demand by
using a Reynolds style interpreter, which can model the full range, from
fully strict to fully lazy languages. We used a monadic style to get fine control
over demand, and to encapsulate the various effects of Haskell.

The present work is the first formal treatment of the fine control of demand in Haskell. We
present a language definition that clearly separates lazy and normal patterns. The patterns
considered were nested patterns, and we did not resort to pattern-match compilation to simplify
the task. The work clearly defines the interaction between {\tt data} (with and without
strictness annotations) and {\tt newtype} data constructors with Haskell's other features. This
definition lays the foundation for an equational reasoning system about real Haskell programs.
The code presented in this paper is available online at {\tt www.cse.ogi.edu/\verb+~+sheard/}.

\bibliographystyle{entcs}
\bibliography{body}  


\end{document}
