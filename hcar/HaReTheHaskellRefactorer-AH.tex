% HaReTheHaskellRefactorer-AH.tex
\begin{hcarentry}[updated]{HaRe --- The Haskell Refactorer}
\label{hare}
\report{Alan Zimmerman}%11/14
\participants{Francisco Soares, Chris Brown, Stephen Adams, Huiqing Li}
\makeheader

Refactorings are source-to-source program transformations which change
program structure and organization, but not program functionality.
Documented in catalogs and supported by tools, refactoring provides
the means to adapt and improve the design of existing code, and has
thus enabled the trend towards modern agile software development
processes.

Our project, \emph{Refactoring Functional Programs}, has as its major
goal to build a tool to support refactorings in Haskell. The HaRe tool
is now in its seventh major release. HaRe supports full Haskell 2010,
and is integrated with (X)Emacs. All the refactorings that HaRe
supports, including renaming, scope change, generalization and a
number of others, are \emph{module-aware}, so that a change will be
reflected in all the modules in a project, rather than just in the
module where the change is initiated.

Snapshots of HaRe are available from our GitHub repository (see below)
and Hackage. There are related presentations and publications from the
group (including LDTA'05, TFP'05, SCAM'06, PEPM'08, PEPM'10, TFP'10,
Huiqing's PhD thesis and Chris's PhD thesis). The final report for the
project appears on the University of Kent Refactoring Functional
Programs page (see below).

There is also a Google+ community called HaRe, and an IRC channel on
freenode called \#haskell-refactorer.

Currently HaRe only supports GHC 7.4.x and 7.6.x. The changes for GHC
7.8.3 showed up the brittleness of the token management process.

As a result, the focus over the last few months has been on improving
the support in the GHC API for this. This work is summarized here
\url{https://ghc.haskell.org/trac/ghc/wiki/GhcApi}, but
the lightning summary is

\begin{compactitem}

\item More parser entry points, to allow parsing fragments.
\item Removal of landmines from the AST, allowing traversals with
  freedom.
\item The parser can now produce annotations, which can be used to
  reproduce the original source from the AST and the annotations only.

\end{compactitem}

The engine for reproducing the source is here \url{https://github.com/alanz/ghc-exactprint/tree/ghc-7.9}.

\subsubsection*{Recent developments}
\begin{compactitem}

\item HaRe 0.7, which is a major change from 0.6 as it makes use of
  the GHC library for analysis, has been released; HaRe 0.7 is
  available on Hackage, and also downloadable from our GitHub page

\item HaRe 0.7 is alpha software, and comes with a limited number of
  refactorings, as the work so far has concentrated on getting the new
  architecture in place to make use of the GHC AST. The new
  architecture has stabilised and the token management while
  manipulating the AST is able to preserve layout, thus maintaining
  the original layout as well as syntactically correct alignment as
  new elements are added or have their size changed.

\item There is plenty to do, so anyone who has an interest is welcome
  to fork the repo and get stuck in.
\item Stephen Adams is continuing his PhD at the University of Kent and
  will be working on data refactoring in Haskell.
\end{compactitem}

\FurtherReading
\begin{compactitem}
\item \url{http://www.cs.kent.ac.uk/projects/refactor-fp/}
\item \url{https://github.com/alanz/HaRe}
\end{compactitem}
\end{hcarentry}
