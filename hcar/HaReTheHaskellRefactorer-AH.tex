% HaReTheHaskellRefactorer-AH.tex
\documentclass[DIV16,twocolumn,10pt]{scrreprt}
\usepackage{paralist}
\usepackage{graphicx}
\usepackage[final]{hcar}

\begin{document}

\begin{hcarentry}[updated]{HaRe --- The Haskell Refactorer}
\label{hare}
\report{Alan Zimmerman}%05/15
\participants{Francisco Soares, Chris Brown, Stephen Adams, Huiqing Li, Matthew
  Pickering, Gracjan Polak}
\makeheader

Refactorings are source-to-source program transformations which change
program structure and organization, but not program functionality.
Documented in catalogs and supported by tools, refactoring provides
the means to adapt and improve the design of existing code, and has
thus enabled the trend towards modern agile software development
processes.

Our project, \emph{Refactoring Functional Programs}, has as its major
goal to build a tool to support refactorings in Haskell. The \emph{HaRe} tool
is now in its eighth major release. \emph{HaRe} supports full Haskell 2010,
and is integrated with (X)Emacs. All the refactorings that \emph{HaRe}
supports, including renaming, scope change, generalization and a
number of others, are \emph{module-aware}, so that a change will be
reflected in all the modules in a project, rather than just in the
module where the change is initiated.

Snapshots of \emph{HaRe} are available from our GitHub repository (see below)
and Hackage. There are related presentations and publications from the
group (including LDTA'05, TFP'05, SCAM'06, PEPM'08, PEPM'10, TFP'10,
Huiqing's PhD thesis and Chris's PhD thesis). The final report for the
project appears on the University of Kent Refactoring Functional
Programs page (see below).

There is a Google Group called
\url{https://groups.google.com/forum/#!forum/hare} and an IRC channel on
freenode called \emph{\#haskell-refactorer}. IRC is the preferred contact method.

Current version of \emph{HaRe} supports 7.10.2 and work is continuing to support GHC version 8.x forward.
The new version makes use of \emph{ghc-exactprint} library, which
only has GHC support from GHC 7.10.2 onwards. 

Development on the core \emph{HaRe} is speeding up at the moment, the focus is on
making sure that deficiencies identified in the API Annotations in GHC used by
\emph{ghc-exactprint} are removed in time for GHC 8.0.1, so that the identity
refactoring can cover more of the corner cases.

There is also a new \emph{haskell-ide} project which will allow \emph{HaRe} to operate
as a plugin and will ease its integration into multiple IDEs.

\subsubsection*{Recent developments}
\begin{compactitem}

\item The current version is 8.2, which supports GHC 7.10.2 only, and was
  released in October 2015.

\item Matthew Pickering has been deeply involved in the \emph{ghc-exactprint}
  development, and successfully completed his Google Summer of Code project
  which involved bringing it up to standard, which has helped tremendously for
  \emph{HaRe}.

\item There is plenty to do, so anyone who has an interest is welcome
  to fork the repo and get stuck in.

\item Stephen Adams is continuing his PhD at the University of Kent and
  will be working on data refactoring in Haskell.
\end{compactitem}

\FurtherReading
\begin{compactitem}
\item \url{http://www.cs.kent.ac.uk/projects/refactor-fp/}
\item \url{https://github.com/RefactoringTools/HaRe}
\item \url{https://github.com/alanz/ghc-exactprint}
\item \url{http://mpickering.github.io/gsoc2015.html}
\item \url{https://github.com/haskell/haskell-ide}
\end{compactitem}
\end{hcarentry}

\end{document}
